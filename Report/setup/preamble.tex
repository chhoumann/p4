\documentclass[11pt,oneside,a4paper]{report}
%%%%%%%%%%%%%%%%%%%%%%%%%%%%%%%%%%%%%%%%%%%%%%%%
% Language, Encoding and Fonts
% http://en.wikibooks.org/wiki/LaTeX/Internationalization
%%%%%%%%%%%%%%%%%%%%%%%%%%%%%%%%%%%%%%%%%%%%%%%%
% Select encoding of your inputs. Depends on
% your operating system and its default input
% encoding. Typically, you should use
%   Linux  : utf8 (most modern Linux distributions)
%            latin1 
%   Windows: ansinew
%            latin1 (works in most cases)
%   Mac    : applemac
% Notice that you can manually change the input
% encoding of your files by selecting "save as"
% an select the desired input encoding. 
\usepackage[utf8]{inputenc}
% Make latex understand and use the typographic
% rules of the language used in the document.
\usepackage{csquotes}
\usepackage[danish,english]{babel}
% Use the palatino font
\usepackage[sc]{mathpazo}
% Set space between lines
\usepackage{setspace}
% Choose the font encoding
\usepackage[T1]{fontenc}
% Remove page numbers from toc
\usepackage{tocloft}
% Itemize customization
\usepackage{enumitem}
% Appendix environment
% \usepackage[title,titletoc]{appendix}

%%%%%%%%%%%%%%%%%%%%%%%%%%%%%%%%%%%%%%%%%%%%%%%%
% Graphics and Tables
% http://en.wikibooks.org/wiki/LaTeX/Importing_Graphics
% http://en.wikibooks.org/wiki/LaTeX/Tables
% http://en.wikibooks.org/wiki/LaTeX/Colors
%%%%%%%%%%%%%%%%%%%%%%%%%%%%%%%%%%%%%%%%%%%%%%%%
% load a colour package
\usepackage{xcolor}
\definecolor{aaublue}{RGB}{33,26,82}% dark blue
% The standard graphics inclusion package
\usepackage{graphicx}
% Set up how figure and table captions are displayed
\usepackage{caption}
\captionsetup{%
  font=footnotesize,% set font size to footnotesize
  labelfont=bf % bold label (e.g., Figure 3.2) font
}
% Adds \HUGE and \ssmall. The latter fills the gap between \scriptsize and \tiny.
\usepackage{moresize}
% Make the standard latex tables look so much better
% \usepackage{array,booktabs}
% ...or you can use this package instead
\usepackage{tcolorbox}
% Define a basic grey box with three custom params.
\newtcolorbox[auto counter,number within=section]{greybox}[3][]{
    title=#3~\thetcbcounter: #2,#1,title filled
}
% Enable the [H] option for figures and tables
\usepackage{float}
% Enable the use of frames around, e.g., theorems
\usepackage{framed}
% Adds support for full page background picture
\usepackage[contents={},color=gray]{background}
%\usepackage[contents=draft,color=gray]{background}

%%%%%%%%%%%%%%%%%%%%%%%%%%%%%%%%%%%%%%%%%%%%%%%%
% Mathematics
% http://en.wikibooks.org/wiki/LaTeX/Mathematics
%%%%%%%%%%%%%%%%%%%%%%%%%%%%%%%%%%%%%%%%%%%%%%%%
% Defines new environments such as equation,
% align and split 
\usepackage{amsmath}
% Adds new math symbols
\usepackage{amssymb}
% Use theorems in your document
% The ntheorem package is also used for the example environment
% When using thmmarks, amsmath must be an option as well. Otherwise \eqref doesn't work anymore.
\usepackage[framed,amsmath,thmmarks]{ntheorem}

%%%%%%%%%%%%%%%%%%%%%%%%%%%%%%%%%%%%%%%%%%%%%%%%
% Listings (Code Snippets)
% https://en.wikibooks.org/wiki/LaTeX/Source_Code_Listings
%%%%%%%%%%%%%%%%%%%%%%%%%%%%%%%%%%%%%%%%%%%%%%%%
\usepackage{listings}
\usepackage{color}

\renewcommand{\lstlistingname}{Code Snippet} % Listing -> Code Snippet
\definecolor{lighter-gray}{RGB}{240,240,240}

\lstset{
  backgroundcolor=\color{lighter-gray},
  extendedchars=true,
  basicstyle=\footnotesize\ttfamily,
  showstringspaces=false,
  showspaces=false,
  numbers=left,
  tabsize=4,
  breaklines=true,
  showtabs=false,
  captionpos=b,
  numberstyle=\footnotesize,
  numbersep=5pt
}

% Define C# as a snippet language.
\usepackage{courier}

\definecolor{Green}{rgb}{0, 0.3, 0}
\definecolor{DarkCyan}{rgb}{0, 0.545, 0.545}
\definecolor{Navy}{rgb}{0, 0, 0.5}
\definecolor{Teal}{rgb}{0, 0.5, 0.5}
\definecolor{DarkGray}{gray}{0.66}
\definecolor{Olive}{rgb}{0.5, 0.5, 0}
\definecolor{Pink}{rgb}{1.0, 0.75, 0.8}
\definecolor{DeepPink}{rgb}{1, 0.08, 0.58}
\definecolor{Brown}{rgb}{0.65, 0.165, 0.165}
\definecolor{DarkViolet}{rgb}{0.58, 0, 0.83}
\definecolor{SaddleBrown}{rgb}{0.55, 0.27, 0.07}
\lstdefinelanguage{CSharp}{
  morecomment = [l]{//}, 
  morecomment = [l]{///},
  morecomment = [s]{/*}{*/},
  morestring=[b]", 
  morestring=[b]',
  basicstyle=\footnotesize\ttfamily,
  commentstyle=\color{Green}\textit,
  stringstyle=\color{blue},
  sensitive = true,
  morekeywords=[1]{this, base},
  keywordstyle=[1]\bfseries,
  morekeywords=[2]{as, is, new, sizeof, typeof, true, false, stackalloc},
  keywordstyle=[2]\color{DarkCyan}\bfseries,
  morekeywords=[3]{else, if, switch, case, default,
  do, for, foreach, while, in},
  keywordstyle=[3]\color{blue}\bfseries ,
  morekeywords=[4]{break, continue, goto, return,
  yield, partial, global, where},
  keywordstyle=[4]\color{Navy},
  morekeywords=[5]{try, throw, catch, finally},
  keywordstyle=[5]\color{Teal}\bfseries,
  morekeywords=[6]{checked, unchecked},
  keywordstyle=[6]\color{DarkGray}\bfseries,
  morekeywords=[7]{fixed, unsafe},
  keywordstyle=[7]\color{Olive},
  morekeywords=[8]{bool, byte, sbyte, char, short, ushort, int, uint, long, ulong, float,
  double, decimal, enum, struct},
  keywordstyle=[8]\bfseries\color{red},
  morekeywords=[9]{class, interface, delegate, object, string,
  void},
  keywordstyle=[9]\color{red},
  morekeywords=[10]{explicit, implicit, operator},
  keywordstyle=[10]\color{Pink}\bfseries,
  morekeywords=[11]{params, ref, out},
  keywordstyle=[11]\bfseries\color{DeepPink},
  morekeywords=[12]{private, protected, internal, public},
  keywordstyle=[12]\bfseries\color{blue},
  morekeywords=[13]{abstract, const, event, var, override, virtual, volatile, extern, readonly, sealed, static},
  keywordstyle=[13]\color{Brown},
  morekeywords=[14]{namespace, using},
  keywordstyle=[14]\bfseries\color{Green},
  morekeywords=[15]{lock},
  keywordstyle=[15]\color{DarkViolet},
  morekeywords=[16]{get, set, add, remove},
  keywordstyle=[16]\color{SaddleBrown},
  morekeywords=[17]{null, value},
  keywordstyle=[17]\bfseries,
}

%%%%%%%%%%%%%%%%%%%%%%%%%%%%%%%%%%%%%%%%%%%%%%%%
% Page Layout
% http://en.wikibooks.org/wiki/LaTeX/Page_Layout
%%%%%%%%%%%%%%%%%%%%%%%%%%%%%%%%%%%%%%%%%%%%%%%%
% Change margins, papersize, etc of the document
\usepackage[
    top    = 3.50cm,
    bottom = 3.50cm,
    left   = 2.50cm,
    right  = 2.50cm
]{geometry}
% Modify how \chapter, \section, etc. look
% The titlesec package is very configureable
\usepackage{titlesec}
\titleformat{\chapter}{\normalfont\Huge\bfseries}{\thechapter}{20pt}{\Huge}
\titlespacing*{\chapter}{0pt}{3.5ex plus 1ex minus .2ex}{2.3ex plus .2ex}%less spacing above chapter title
\titleformat*{\section}{\normalfont\Large\bfseries}
\titleformat*{\subsection}{\normalfont\large\bfseries}
\titleformat*{\subsubsection}{\normalfont\normalsize\bfseries}
%\titleformat*{\paragraph}{\normalfont\normalsize\bfseries}
%\titleformat*{\subparagraph}{\normalfont\normalsize\bfseries}
%\setlength{\parindent}{0pt}%remove indentation on newline

% Change the headers and footers
\usepackage{fancyhdr}
\pagestyle{fancy}
\fancyhf{} %delete everything
\renewcommand{\headrulewidth}{0pt} %remove the horizontal line in the header
\setlength{\headheight}{14pt}
\renewcommand{\chaptermark}[1]{\markboth{\MakeUppercase{\ \thechapter.\ #1}}{}}%removes "Chapter" from \leftmark
% Define headers
\lhead{\small\nouppercase\leftmark} %left side of head
\rhead{\small\nouppercase\rightmark} %right side of head
% Define footers
\newcommand{\rightfoot}{\small Page \textbf{\thepage} of \textbf{\pageref{bib:mybiblio}}}
\newcommand{\leftfoot}{\small\authorname}
\rfoot{\rightfoot}%page number on all pages
\lfoot{\leftfoot}%group name on all pages
% Redefine the plain page style to modify footers on chapter pages
\fancypagestyle{plain}{%
    \fancyhf{} %delete everything
    \rfoot{\rightfoot} %page number on all chapter pages
    \lfoot{\leftfoot}%group name on all chapter pages
}

% Applies settings to appendix environment
\newcommand{\applyappendixconfig}{
    \addcontentsline{toc}{chapter}{Appendices}
    \renewcommand{\thesection}{\Alph{section}}
    % Hide page numbering in toc
    \addtocontents{toc}{\cftpagenumbersoff{chapter}}
    \addtocontents{toc}{\cftpagenumbersoff{section}}
    \addtocontents{toc}{\cftpagenumbersoff{subsection}}
    \addtocontents{toc}{\cftpagenumbersoff{subsubsection}}
    % Add appendices chapter without chapter number
    \chapter*{Appendices}
    \phantomsection
    % Remove head and foot from appendices
    \emptyheadfoot % Custom command to make head and foot empty
}

% Remove head and foot (Used in \applyappendixconfig)
\newcommand{\emptyheadfoot}{
    \rhead{ }
    \lhead{ }
    \rfoot{ }
    \lfoot{ }
    % Redefine the plain page style to modify footers on chapter pages
    \fancypagestyle{plain}{
        \fancyhf{} %delete everything
        \rfoot{ }
        \lfoot{ }
    }
}

% Do not stretch the content of a page. Instead,
% insert white space at the bottom of the page
\raggedbottom
% Enable arithmetics with length. Useful when
% typesetting the layout.
\usepackage{calc}

%%%%%%%%%%%%%%%%%%%%%%%%%%%%%%%%%%%%%%%%%%%%%%%%
% Bibliography
% http://en.wikibooks.org/wiki/LaTeX/Bibliography_Management
%%%%%%%%%%%%%%%%%%%%%%%%%%%%%%%%%%%%%%%%%%%%%%%%
\usepackage[backend=bibtex,
  bibencoding=utf8,
  style=ieee
  ]{biblatex}
\addbibresource{bib/bibliography}

%%%%%%%%%%%%%%%%%%%%%%%%%%%%%%%%%%%%%%%%%%%%%%%%
% Misc
%%%%%%%%%%%%%%%%%%%%%%%%%%%%%%%%%%%%%%%%%%%%%%%%
% Add bibliography and index to the table of contents
\usepackage[nottoc]{tocbibind}
% Used to place fixme/todo notes as notes in the footer
% Full documentation: https://www.lrde.epita.fr/~didier/software/fixme.pdf
\usepackage[footnote,english,draft,silent,nomargin]{fixme}
% Mock text
\usepackage{lipsum}

%%%%%%%%%%%%%%%%%%%%%%%%%%%%%%%%%%%%%%%%%%%%%%%%
% Hyperlinks
% http://en.wikibooks.org/wiki/LaTeX/Hyperlinks
%%%%%%%%%%%%%%%%%%%%%%%%%%%%%%%%%%%%%%%%%%%%%%%%
% Enable hyperlinks and insert info into the pdf
% file. Hypperref should be loaded as one of the 
% last packages
\usepackage{hyperref}
\hypersetup{%
% 	pdfpagelabels=true,%
	plainpages=false,%
	pdfauthor={Author(s)},%
	pdftitle={Title},%
	pdfsubject={Subject},%
	bookmarksnumbered=true,%
	colorlinks=false,%
	citecolor=black,%
	filecolor=black,%
	linkcolor=black,% you should probably change this to black before printing
	urlcolor=black,%
	pdfstartview=FitH%
}

%%%%%%%%%%%%%%%%%%%%%%%%%%%%%%%%%%%%%%%%%%%%%%%%
% Hyphenation
% http://en.wikibooks.org/wiki/LaTeX/Formatting#Hyphenation
%%%%%%%%%%%%%%%%%%%%%%%%%%%%%%%%%%%%%%%%%%%%%%%%
\hyphenation{ex-am-ple hy-phen-a-tion short}
\hyphenation{long la-tex}
