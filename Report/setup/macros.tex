%%%%%%%%%%%%%%%%%%%%%%%%% Setup %%%%%%%%%%%%%%%%%%%%%%%%%
\newcommand{\authorname}{Group SW422F21}%name of author. Shown on the title page and in the footer.
\setcounter{tocdepth}{2} % Depth of the ToC. 2 will result in subsections being included.
\linespread{1.2} % Line spacing

%%%%%%%%%%%%%%% Project Specific Commands %%%%%%%%%%%%%%%
\newcommand{\frontend}{front end}
\newcommand{\Frontend}{Front end}

\newcommand{\backend}{back end}
\newcommand{\Backend}{Back end}

\newcommand{\ui}{user interface}
\newcommand{\Ui}{User interface}

\newcommand{\dazel}{Dazel}
\newcommand{\abstractsemanticclass}{\texttt{EnvironmentStore}}

\newcommand{\sql}{SQL}
\newcommand{\sqldb}{\sql-database}

\newcommand{\envs}{env_v, sto}

%%%%%%%%%%%%%%%% General Custom Commands %%%%%%%%%%%%%%%%
% Insert figures easily (Uses the file name as the label):
% \fig{SIZE (decimal)}{FILE NAME}{CAPTION}
\newcommand{\fig}[3]{
	\begin{figure}[H] % Alternatively [htbp] 
	    \centering
		\includegraphics[width=#1\textwidth]{figures/#2}
		\caption{#3}
		\label{fig:#2} % Note that "fig:" is automatically added to the label here
	\end{figure} 
}

% Insert a definition boxes easily with the box command:
% \dbox{TITLE}{box:LABEL}{CONTENT}
\newcommand{\dbox}[3]{
    \vspace*{0.5cm}
    \begin{greybox}[label={#2}]{#1}{Definition~box}
        #3
    \end{greybox}
    \vspace*{0.5cm}
}

% Quote a source with the excerpt command
% \excerpt{CONTENT}{AUTHOR NAME \cite{TAG}}
\newcommand{\excerpt}[2]{
\begin{quote}
    \textit{#1}
\end{quote}
\begin{center}
    #2
\end{center}
}

% Insert use cases easily with the \usecase command:
% \usecase{TITLE}{uc:LABEL}{USE CASE}{OBJECTS}{FUNCTIONS}{CAPTION}
\newenvironment{usecaseenv}{
    \def\arraystretch{2}
    \begin{tabular}{lp{11cm}}\hline
}{
    \hline\end{tabular}
    \def\arraystretch{1}
}

\newcommand\addheading[1]{
    \multicolumn{2}{c}{\textbf{\textit{#1}}}\\ \hline
}
\newcommand\addrow[2]{\textbf{#1}\begin{minipage}[t][][t]{11cm} \end{minipage}% 
   &\begin{minipage}[t][][t]{11cm}
    #2
    \end{minipage}\\
}

% The actual command definition
\let\oldFigureName\figurename %save the old definition of the caption's figure name
\newcommand{\usecase}[6]{
    \vspace*{0.5cm} % adds a bit of padding to make it look nicer
    \renewcommand{\figurename}{Use case} %call figure name "Use case" instead
    \begin{figure}[htbp]
        \begin{center}
            \begin{usecaseenv}
                \addheading{#1} 
                \addrow{Use case:}{#3}
                \addrow{Objects:}{#4}
                \addrow{Functions:}{#5}
            \end{usecaseenv}
        \end{center}
        \caption{#6}
        \label{#2}
    \end{figure}
    \renewcommand{\figurename}{\oldFigureName} %reset caption figure name
}

%%%%%%%%%%%%%%%%%%% EASY REFERENCINGs %%%%%%%%%%%%%%%%%%%
% Avoid using \ref{} to make sure you reference in the same way throughout the report

% Figure (Use file name as label reference)
\newcommand{\figref}[1]{figure \ref{fig:#1}} % "fig:" is added automatically here
\newcommand{\Figref}[1]{Figure \ref{fig:#1}} % "fig:" is added automatically here

% Chapter
\newcommand{\chapref}[1]{chapter \ref{#1}}
\newcommand{\Chapref}[1]{Chapter \ref{#1}}
% Section
\newcommand{\secref}[1]{section \ref{#1}}
\newcommand{\Secref}[1]{Section \ref{#1}}
% Subsection
\newcommand{\subsecref}[1]{subsection \ref{#1}}
\newcommand{\Subsecref}[1]{Subsection \ref{#1}}
% Appendix
\newcommand{\appref}[1]{appendix \ref{#1}}
\newcommand{\Appref}[1]{Appendix \ref{#1}}

% Definition Box
\newcommand{\dboxref}[1]{definition box \ref{#1}}
\newcommand{\Dboxref}[1]{Definition box \ref{#1}}
% Table
\newcommand{\tabref}[1]{table \ref{#1}}
\newcommand{\Tabref}[1]{Table \ref{#1}}
% Code snippet
\newcommand{\snipref}[1]{snippet \ref{#1}}
\newcommand{\Snipref}[1]{Snippet \ref{#1}}
% Use cases
\newcommand{\ucref}[1]{use case \ref{#1}}
\newcommand{\Ucref}[1]{Use case \ref{#1}}
