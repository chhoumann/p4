\subsection*{Member Access Link Checking} \label{sec:Linker}
Since the beginning of the project, one of the largest tasks was figuring out how to connect screens together, both syntacically and semantically. 
As mentioned in chapter \ref{chap:language_design}, we decided to link screens together by allowing access to other screens and their variables through member access. 


Two examples of member access can be seen in \snipref{lst:LinkingExample}. 
On line \ref{line:MemberAccessExample1}, \texttt{Exit1} in \texttt{SampleScreen1} is being connected to \texttt{Exit1} in \texttt{SampleScreen2} through member access in the \texttt{Exit} function invocation.
Similarly, on line \ref{line:MemberAccessExample2}, \texttt{Exit1} in \texttt{SampleScreen2} is being connected to \texttt{Exit1} in \texttt{SampleScreen1}.

\begin{lstlisting}[language=CSharp, caption={\dazel{} source code example of member access.}, label={lst:LinkingExample},escapechar=|]
// ./src/SampleScreen1.txt
Screen SampleScreen1 
{
	Map 
	{
		Size(30, 24);
		Floor("Grass");
	}
	
	Exits 
	{
		Exit1 = Exit([8, 2], SampleScreen2.Exits.Exit1); |\label{line:MemberAccessExample1}|
	}
}

// ./src/SampleScreen2.txt
Screen SampleScreen2
{
	Map 
	{
		Size(40, 40);
		Floor("Stone");
	}
	
	Exits 
	{
		Exit1 = Exit([4, 0], SampleScreen1.Exits.Exit1); |\label{line:MemberAccessExample2}|
	}
}
\end{lstlisting}

It is worth noting that all screens and objects defined within a screen are always accessible from the outside, and that member access is currently restricted to \texttt{Exit} types.


To illustrate how member access is implemented in \dazel{}, we will first highlight the \textbf{LinkChecker} class whose responsibility to check the legality of member access nodes. This is done directly after type checking.
As with previous examples, the \textbf{LinkChecker} class implements a visitor pattern to traverse the AST.
Once the \textbf{LinkChecker} encounters a function invocation with member access through traversal of the AST, it will first verify that the referenced \textbf{GameObject} is valid. 
If the \textbf{GameObject} is valid, it will check if any of the parameters is a member access node and, if so, traverse them. This can be seen in the function invocation visit method on lines \ref{line:VisitFunctionInvocationStart} to \ref{line:VisitFunctionInvocationEnd}. 

\begin{lstlisting}[language=CSharp, caption={linker}, label={lst:LinkerClass},escapechar=|]
public sealed class LinkChecker : ICompleteVisitor
{
	... 

	public void Visit(FunctionInvocationNode functionInvocationNode) |\label{line:VisitFunctionInvocationStart}|
	{
		switch (functionInvocationNode.Function) 
		{
			case ScreenExitFunction screenExitFunction:
			{
				if (!abstractSyntaxTree.TryRetrieveGameObject(screenExitFunction.ConnectedScreenIdentifier))
				{
					DazelLogger.EmitError($"Screen {string.Join(".", screenExitFunction.ConnectedScreenIdentifier)} does not exist", functionInvocationNode.Token);
				}
				break;
			}
			case ExitFunction exitFunction:
				exitFunction.MemberAccessNode.Accept(this);
				break;
			
			...
		}

		foreach (ValueNode valueNode in functionInvocationNode.Parameters)
		{
			if (valueNode is MemberAccessNode memberAccessNode)
			{
				memberAccessNode.Accept(this);
			}
		}
	} |\label{line:VisitFunctionInvocationEnd}|

	...

	public void Visit(ExitValueNode exitValueNode)
	{
		if (exitValueNode is TileExitValueNode tileExit)
		{
			EnvironmentStore.AccessMember(tileExit.ToExit);
		} 

		...
	}	
}
\end{lstlisting}

As previously mentioned, the \abstractsemanticclass{} class stores all the symbol tables for the program. 
At this point in compilation, all symbol tables have been created and populated, making them usable for member access.
The \texttt{AccessMember} method in \snipref{lst:VisitAssignment} takes a list of identifiers which must be the path to a specific identifier in a \textbf{GameObject}.
For example, the previously shown syntatic member access \texttt{"SampleScreen1.Exits.Exit1"} in \snipref{lst:LinkingExample} becomes a list of three strings, 
\texttt{\{"SampleScreen1", "Exits", "Exit1"\}}.

\begin{lstlisting}[language=CSharp, caption={Visit assignment}, label={lst:VisitAssignment},escapechar=|]
public abstract class EnvironmentStore
{
	|\dots|

	private static readonly Dictionary<string, SymbolTable> TopSymbolTables = new Dictionary<string, SymbolTable>();
	
	|\dots|

	public static VariableSymbolTableEntry AccessMember(List<string> identifierList)
	{
		string symbolIdentifier = identifierList[identifierList.Count - 1]; |\label{line:MemberAccessVariableIdentifier}|
		
		SymbolTable symbolTable = TopSymbolTables[identifierList[0]]; |\label{line:MemberAccessGameObjectIdentifier}|
		SymbolTableEntry symbolTableEntry = symbolTable.RetrieveSymbolInChildScope(symbolIdentifier); |\label{line:RetrieveSymbolInChildScope}|

		if (symbolTableEntry is VariableSymbolTableEntry variableSymbolTableEntry)
		{
			return variableSymbolTableEntry;
		}
		
		return null;
	}
}
\end{lstlisting}

As can be seen on line \ref{line:MemberAccessVariableIdentifier}, the last element in the list is assumed to be the identifier of the variable that one wishes access, in this case, it is \texttt{Exit1}.
In addition, the first element in the list is assumed to be the \textbf{GameObject} identifier, in this case, it is \texttt{SampleScreen1}. 
This is shown on line \ref{line:MemberAccessGameObjectIdentifier}.

The elements in the middle is assumed to be the path to the scope in which the variable lives, in this case, we are simply accessing the \texttt{Exits} scope. 
This has not yet been implemented, and so in the current implementation of member access, the \texttt{RetrieveSymbol-
InChildScope} method on line \ref{line:RetrieveSymbolInChildScope} searches through every scope in the \textbf{GameObject} recursively until it finds the variable with the provided identifier, or until it has checked all scopes and found nothing. This is not ideal, as having defined a variable with the same name in two different scopes would simply return the first one that is encountered.


This concludes the semantic analysis phase of the compiler. Next, we will discuss the code generation phase.