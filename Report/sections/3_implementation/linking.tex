\subsection*{Member Access Linking} \label{sec:Linker}
Since the beginning of the project, one of the largest tasks was figuring out how to connect screens together, both syntacically and semantically. 
As mentioned in chapter \ref{chap:language_design}, we decided to link screens together by allowing access to other screens and their variables through member access. 


Two examples of member access can be seen in \snipref{lst:LinkingExample}. 
On line \ref{line:MemberAccessExample1}, \texttt{Exit1} in \texttt{SampleScreen1} is being connected to \texttt{Exit1} in \texttt{SampleScreen2} through member access in the \texttt{Exit} function invocation.
Similarly, on line \ref{line:MemberAccessExample2}, \texttt{Exit1} in \texttt{SampleScreen2} is being connected to \texttt{Exit1} in \texttt{SampleScreen1}.

\begin{lstlisting}[language=CSharp, caption={\dazel{} source code example of member access.}, label={lst:LinkingExample},escapechar=|]
// ./src/SampleScreen1.txt
Screen SampleScreen1 
{
	Map 
	{
		Size(30, 24);
		Floor("Grass");
	}
	
	Exits 
	{
		Exit1 = Exit([8, 2], SampleScreen2.Exits.Exit1); |\label{line:MemberAccessExample1}|
	}
}

// ./src/SampleScreen2.txt
Screen SampleScreen2
{
	Map 
	{
		Size(40, 40);
		Floor("Stone");
	}
	
	Exits 
	{
		Exit1 = Exit([4, 0], SampleScreen1.Exits.Exit1); |\label{line:MemberAccessExample2}|
	}
}
\end{lstlisting}

It is worth noting that all screens and objects defined within a screen are always accessible from the outside, and that member access is restricted to specific types such as the \texttt{Exit} type.
