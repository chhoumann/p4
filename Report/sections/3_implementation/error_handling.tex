\section{Error handling}
It is useful for users to get error messages detailing any syntax errors there might have been found during parsing.

A simple solution is to provide the user with the offending token and/or line. This is very basic error output, which could be augmented by a detailed message telling the user what is wrong. A missing semicolon, for example. Even better if the message told the user how to fix the error. 

Going a level further, the parser could attempt to make a recovery by handling the error, if possible. There are a few ways to do this.

First is panic-mode recovery, in which the parser removes input tokens until a synchronizing token is found. In the case of \dazel{}, this could be a semicolon, indicating the end of a statement. This is a simple solution, but it may skip a considerable amount of input, which may produce unexpected results\cite{crafting_a_compiler}.

It is also possible to do phrase-level recovery. This means performing corrections locally in some erroneous phrase. This could be something as trivial as inserting a missing semicolon, or even deleting an extraneous one. In performing corrections, one must be careful not to enter an infinite loop\cite{spo_course}.

A third option is to augment the grammar with productions to handle common errors\cite{spo_course}. Drawing from example; in C\#, users are prone to attempt using the common \texttt{length} member of arrays on lists, instead of the correct \texttt{count} member. This could be handled by creating an 'alias' for \texttt{length}.

The \dazel{} parser is able to do some basic error recovery due to ANTLR. Namely, \textit{single-token deletion} and \textit{single-token insertion}. If one were to delete the closing curly-bracket, \}, the parser would still be able to recognize the input. Likewise, if one were to add an extraneous closing curly-bracket, the input would still be recognized.

Besides what is provided by ANTLR itself, ANTLR also provides methods of extending the parser to allow for custom recovery options. These were not pursued in the \dazel{} compiler, and as such, it does not support error-recovery at this time. Instead, it leverages multiple interfaces which allows it to communicate detailed errors to the user.

The first interface builds on an ANTLR feature. 

By extending the ANTLR \texttt{BaseErrorListener}, it is possible to extract the offending token and line number when the parser encounters errors. To achieve this, we used the code shown in code \snipref{lst:SyntaxErrorCode}.

\begin{lstlisting}[language=CSharp, caption={The altered \texttt{SyntaxError} method.}, label={lst:SyntaxErrorCode}, escapechar=|]
public override void SyntaxError(TextWriter output, IRecognizer recognizer, 
    IToken offendingSymbol, int line, int charPositionInLine,
    string msg, RecognitionException e)
{
    StringBuilder errorLog = new StringBuilder();
    base.SyntaxError(output, recognizer, offendingSymbol, line, charPositionInLine, msg, e);
    
    // UnderlineError(...) will not be shown in the spirit of brevity.
    string underlineError = UnderlineError(recognizer, offendingSymbol, line, charPositionInLine);

    errorLog.AppendLine($"File: {recognizer.InputStream.SourceName}");
    errorLog.AppendLine($"Line {line}:{charPositionInLine} at {offendingSymbol}: {msg}\n");
    errorLog.AppendLine($"Error: \n{underlineError}");

    DazelLogger.EmitError(errorLog.ToString(), offendingSymbol);|\label{code:dazellogger}|
}
\end{lstlisting}

This leads us to the second interface. \dazel{} uses a custom-built error logging system, which can be seen invoked on line \ref{code:dazellogger}.

The \texttt{DazelLogger} is a static class which allows the \dazel{} compiler to log errors. The class contains three methods, \texttt{EmitError}, \texttt{EmitWarning}, and \texttt{EmitMessage}. These represent the levels of messages that the user may be presented with. Every member of this class is \texttt{static}, such that they can be accessed from anywhere, without instantiating a new logger - and without unnecessary dependency injection.

The most interesting case is \texttt{EmitError}, which can be seen in code \snipref{lst:EmitErrorCode}. It sets the \texttt{HasErrors} flag to true, such that the \dazel{} compiler can react accordingly. Likewise, it formats the output message and invokes an event, passing it in as a parameter.

\begin{lstlisting}[language=CSharp, caption={\texttt{EmitError} in the \texttt{DazelLogger}.}, label={lst:EmitErrorCode}, escapechar=|]
public static void EmitError(string message, IToken token)
{
    HasErrors = true;
    
    string output = $"Error on line {token.Line} in {token.InputStream.SourceName}:\n{message}";
    LogMessageReceived?.Invoke(output, LogType.Error);|\label{code:LogMessageReceived}|

    ...
}
\end{lstlisting}

The \texttt{LogMessageReceived} event, seen on line \ref{code:LogMessageReceived}, is written to a console GUI in the \dazel{} game. An example can be seen in \figref{dazelconsole}, showing both warnings and messages. The messages shown in this particular example represent usages of \dazel{}'s \texttt{Print} function. In another example, seen in \figref{dazelconsoleerror}, an error is displayed. This error is due to writing "this is an error" in a \texttt{GameObjectContent} block. Lastly, \figref{dazel_console_custom_error} shows the \dazel{} console with a type error. The difference between the latter two errors is that syntax errors are picked up by the parser, and errors in semantics are picked up during later phases. Both provide different kinds of error messages, indicating what went wrong.

\fig{1}{dazel_console}{The \dazel{} console with no error.\label{fig:dazelconsole}}
\fig{1}{dazel_console_error}{The \dazel{} console with a syntax error.\label{fig:dazelconsoleerror}}
\fig{1}{dazel_console_custom_error}{The \dazel{} console with a type error.\label{fig:dazel_console_custom_error}}

Having proper error output in a compiler is, as previously mentioned, very important. \dazel{} has detection of errors in multiple phases, which allows it to give a more detailed and descriptive error message. A clear next step would be to implement error recovery, as described earlier in this section.

The next chapter details how we have ensured that \dazel{} runs as it should - through testing.
