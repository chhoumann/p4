\section{Scanner} \label{chap:scanner}

The first phase in a compiler is the scanner, also called "lexical analysis" or "lexer".
The purpose of the scanner is to read some sourcecode and differentiate the different words, by combining 
characters into \emph{tokens}.
A \emph{token} is a group of characters that form a unit. Examples of tokens can be 
\emph{identifiers}, \emph{numbers} and \emph{special symbols}. 
These tokens are idenified by their \emph{lexeme}, which is a sequence of characters that matches the pattern of a token.
For example, the number "9" written
in the sourcecode, can be considered a number token, and the lexeme for it would the character "9" itself. 
The pattern of a token is specified by \emph{regular expression}.
\emph{Regular expression} is a way to describe a search pattern, which works by trying to match an input
text with the \emph{regular expression} \cite{crafting_a_compiler}.

The scanner works by taking in some sourcecode as input, and then scanning it character for character, 
ignoring any white space or comments. It will then identify the lexemes that makes up a token
based on the regular expression that has been specified for each token category. If it fails to recognise 
a token or finds it invalid, it will create an error. If the tokens are valid however, they
will be passed to the parser when necessary. 






