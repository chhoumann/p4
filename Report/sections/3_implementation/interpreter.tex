\section{The Game} \label{sec:Interpreter}
Following the code generation phase is the final result of the code compilation — the game itself. 
The game is built using the Unity engine. It interprets the intermediate models in order to build the world defined by the user in the \dazel{} source code. As can be seen in \figref{Dazel_Title}, the user is presented with a few options when launching the application.

\fig{0.8}{Dazel_Title}{The title screen of Dazel. Grass graphics made by Bonsaiheldin\cite{bonsaiheldin_grass_2016}.}
 
The ``Play'' button runs the \dazel{} compiler and, if compilation succeeded, runs the interpreter that builds the game from the intermediate models. 


The ``Enable logging'' checkbox toggles the logging of messages, warnings and errors in the console GUI. We will elaborate on error handling in \secref{ErrorHandling}.


The ``Open Working Directory'' button reveals the working directory in the operating systems' file explorer.
Inside the working directory, two sub-directories are located: the ``src'' directory, in which the user creates text files with source code, and the ``gfx'' directory in which the user places graphic assets. Inside this directory, the user can either place image files and reference them directly with a string like ``Grass.png.'' However, one may also create directories inside the ``gfx'' directory and place multiple different images files inside. This way, the user can, for example, create a directory named ``Grass'' with three different grass textures inside, and by calling \texttt{Floor("Grass")}, the interpreter assumes that the lack of a file extension means that it should search for a directory with the name ``Grass''. If found, a random image file is picked from within for each floor tile in the given screen. This allows the user to easily create more organic looking worlds.


To illustrate this, consider the \dazel{} code shown in \snipref{lst:DazelGrassExample}.
\begin{lstlisting}[language=CSharp, caption={}, label={lst:DazelGrassExample},escapechar=|]
Screen SampleScreen1 
{
	Map 
	{
		Size(24, 24);
		Floor("Grass");
	}
}
\end{lstlisting}

Creating a directory called ``Grass'' containing multiple different image files shown in \figref{Dazel_Random_Grass2} will then produce the result shown in \figref{Dazel_Random_Grass1}.

\fig{0.6}{Dazel_Random_Grass2}{Using a directory called ``Grass'' in the ``gfx'' folder to store multiple different grass textures.}

\fig{0.6}{Dazel_Random_Grass1}{The result of the code in \snipref{lst:DazelGrassExample} with the directory setup shown in \figref{Dazel_Random_Grass2}.}

The game uses 16 pixels per unit, meaning each tile used in the game should be a $16\times 16$ image. Using a larger image is possible, though the result may be undesirable. Entities, however, may be as big as the user prefers. 
Although trivial to implement, we decided to not throw errors when images with odd sizes are used. We feel that a large part of learning game development and programming is experimentation and seeing what happens when trying out strange ideas.


As previously mentioned, one of the most important aspect of \dazel{} is linking screens together to create a large world. Consider the code creating two different screens shown in \snipref{lst:DazelScreenExit}.

\begin{lstlisting}[language=CSharp, caption={}, label={lst:DazelScreenExit},escapechar=|]
// ./src/SampleScreen1.txt
Screen SampleScreen1 
{
	Map 
	{
		Size(24, 24);
		Floor("Grass");
	}

	Exits
	{
		ScreenExit(Left, SampleScreen2); |\label{line:ScreenExitExampleCode}|
	}
}

// ./src/SampleScreen2.txt
Screen SampleScreen2
{
	Map 
	{
		Size(24, 24);
		Floor("Stone");
	}
}
\end{lstlisting}

On line \ref{line:ScreenExitExampleCode}, the \texttt{ScreenExit} function is invoked, linking the two screens together.
By exiting to the left of \texttt{SampleScreen1}, the player will enter \texttt{SampleScreen2}.
This is illustrated in \figref{Dazel_ScreenExit}.

\fig{0.8}{Dazel_ScreenExit}{The result of the code in \snipref{lst:DazelScreenExit}.
Stone graphics made by KennyNL\cite{noauthor_kenney_nodate}.}

The generation and connecting of screens is done inside a Unity \texttt{MonoBehaviour}\footnote{All components in Unity inherit from \texttt{MonoBehaviour}. This differentiates them from plain C\# classes and provides them with access to engine-specific behavior.} class called \texttt{World}.
This class is responsible for taking the \texttt{ScreenModels} generated in the code generation phase and transforming it into interconnected screens which are then populated by entities. 
As this interpretation is based on the Unity game engine and the focus of this report is the \dazel{} language, 
\snipref{lst:WorldClassUnity} shows a heavily simplified version of the \texttt{World} class that illustrates its structure without delving into the implementation details.

\begin{lstlisting}[language=CSharp, caption={The \texttt{World} class responsible for generating and connectind screens.}, label={lst:WorldClassUnity},escapechar=|]
public sealed class World : MonoBehaviour
{
	...
    
    public static IEnumerable<ScreenModel> ScreenModels { get; set; } |\label{line:ScreenModelsIEnumerable}|
    public static GameScreen CurrentScreen { get; private set; }
    
    private void Awake()
    {
        if (ScreenModels == null) return;
        
        Dictionary<string, GameScreen> screens = CreateScreens();

        ConnectScreens(screens);
        SpawnEntities(screens);
    }

	...

    private Dictionary<string, GameScreen> CreateScreens()
    {
        Dictionary<string, GameScreen> screens = new 
            Dictionary<string, GameScreen>();

        foreach (ScreenModel screenModel in ScreenModels) |\label{line:ScreenModelsIteration1}|
        {
			...
        }

        return screens;
    }

    private static void ConnectScreens(
        IReadOnlyDictionary<string, GameScreen> screens)
    {
        foreach (ScreenModel screenModel in ScreenModels) 
        {
            GameScreen connectedScreen = 
                screens[screenExitModel.ConnectedScreenIdentifier];

            foreach (ScreenExitModel screenExitModel in screenModel.ScreenExits)
            {
				...
            }
        }
    }

    private void SpawnEntities(IReadOnlyDictionary<string, GameScreen> screens)
    {
        const int ppu = GameManager.PixelsPerUnit;
        
        foreach (ScreenModel screenModel in ScreenModels)
        {
            foreach (EntityModel entityModel in screenModel.Entities)
            {
				...
            }
        }
    }
	
	...
}
\end{lstlisting}

As can be seen on line \ref{line:ScreenModelsIEnumerable}, the class stores a reference to the \texttt{ScreenModels} from the code generation phase in the compiler and uses the models to build the game world. 
Entities may also be spawned using the previously described \texttt{SpawnEntities} method, however due to time constraints, entities are currently static objects that the player can simply collide with. 
In addition, the player is a pre-defined asset with animations for walking around the game world. We have discussed allowing the user to create the player themselves, though this was not prioritized.


It is worth noting that the interpreter that builds the game does not implement game behavior for large parts of the \dazel{} language such as the \texttt{Walls}, \texttt{Square} and \texttt{Line} functions. 
This is because its purpose is to serve as a proof of concept that gives an idea of what the final result would look like. 
For this project, we have focused almost exclusively on the design and implementation of the \dazel{} language, meaning the game itself was never a priority.
