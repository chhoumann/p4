\section{Design evaluation} \label{chap:design evaluation}
When desiging a programming language, it is important to consider a set of essential criteria to evaluate the different aspects of the language.
\Figref{Design_Criteria} from \cite{sebesta_concepts_2016} covers the most important of such criteria.

\fig{0.8}{Design_Criteria}{Design criteria for a progamming language.}

All these criteria cannot be fulfiled at once as they affect each other; rather, it is about prioritizing what matters for a given language. 
To understand the implications of these criteria, we will now describe them.

\textit{Readability} refers to how easy it is to read a language.
The more readable a language is, the easier it is to learn.

\textit{Writability} of a language is how easy it is to write programs using it.
This is important to note that writability must be seen within the context of the language's target domain and should not necessarily be compared to other languages.

\textit{Reliability} is achieved if a language performs as specified under all circumstances.

\textit{Simplicity} describes how easy the language is to read and understand. 
A large number of constructs and features in a language will harm its simplicity and make it more diifficult to learn.

\textit{Orthogonality} describes how many different ways primitive constructs can be combined. 
High orthogonality means that primitive constructs in a language can be combined in many different ways. 
Orthogonality impacts simplicity; the more possible combinations, the simpler the language is to learn, read and understand.

\textit{Data types} represent the different types that can be worked with in the language. 
For example, including a boolean data type makes boolean expressions more readable as one can use the keywords \texttt{true} and \texttt{false} as opposed to numeric values like 0 and 1. 

\textit{Syntax design} is the design of the elements in the language as well as its form. 
This is largely subjective and based on preference, and has a large impact on the readability and writability of the language. 
The words chosen to represent the syntax should imply their purpose. 

\textit{Abstraction support} means allowing reuse of code in the language itself. 
This can be achieved through constructs such as functions and classes.

\textit{Expressivity} refers to how one can express different operations in the language.
Expressivity usually means that languages have convenient ways of specifying computations. 
Increasing expressivity also increases writability of the language, though it may decrease readability as more operations harm a language's simplicity.
 
\textit{Type checking} concerns itself with checking of type errors either at compile time or run time. 
It is often preferable to do type checking at compile time as it is cheaper and it increases the reliability.

\textit{Exception handling} refers to the interception of run-time errors at which point correct measure must be taken. 
Attempting to index an array out of bounds is an example where proper exception handling is desirable.
Some languages allow this, which leads to undefined behavior, whereas other languages terminate the process.
Good exception handling increases a language's reliability.

\textit{Aliasing} is the concept of having multiple distinct names that refer to the same memory cell. 
It is typically considered dangerous practice for programming languages, as more responsibility is put on the programmer to maintain memory address references\secref{sec:scientific_research}.

For \dazel{}, the most important factors are readability, writability, simplicity and syntax design. 
These will be  prioritized as we believe they have the largest influence on how easy it is to learn a language.
This assumption is based on the research papers described in \secref{sec:scientific_research}.

In the following sections, we will introduce the constructs featured in the design of \dazel{}.