\section{Design evaluation} \label{chap:design evaluation}
When desiging a programming language, it is important to consider a set of essential criteria to evaluate the different aspects of the language.
\Figref{Design_Criteria} from \cite{sebesta_concepts_2016} covers the most important of such criteria.

\fig{0.8}{Design_Criteria}{Design criteria for a progamming language.}

All these criteria cannot be fulfiled at once as they affect each other; rather, it is about prioritizing what matters for a given language. 
To understand the implications of these criteria, we will now describe them.

\textit{Simplicity} describes how easy the language is to read and understand. 
A large number of constructs and features in a language will harm its simplicity and make it more diifficult to learn.

\textit{Orthogonality} describes how many different ways primitive constructs can be combined. 
High orthogonality means that primitive constructs in a language can be combined in many different ways. 
Orthogonality impacts simplicity; the more possible combinations, the simpler the language is to learn, read and understand.

\textit{Data types} represent the different types that can be worked with in the language. 
For example, including a boolean data type makes boolean expressions more readable as one can use the keywords \texttt{true} and \texttt{false} as opposed to numeric values like 0 and 1. 

\textit{Syntax design} is the design of the elements in the language as well as its form. 
This is largely subjective and based on preference, and has a large impact on the readability and writability of the language. 
As described in \secref{sec:scientific_research}, the words chosen to represent the syntax should imply their purpose.

\textit{Abstraction support} means allowing reuse of code in the language itself. 
This can be achieved through constructs such as functions and classes.

\textit{Expressivity} refers to how one can express different operations in the language.
Expressivity usually means that languages have convenient ways of specifying computations. 
Increasing expressivity also increases writability of the language, though it may decrease readability as more operations harm a language's simplicity.
 
\textit{Type checking} concerns itself with checking of type errors either at compile time or run time. 
It is often preferable to do type checking at compile time as it is cheaper and it increases the reliability.

\textit{Exception handling} refers to the interception of run-time errors at which point correct measure must be taken. 
Attempting to index an array out of bounds is an example where proper exception handling is desirable.
Some languages allow this, which leads to undefined behavior, whereas other languages terminate the process.
Good exception handling increases a language's reliability.

\textit{Aliasing} is the concept of having multiple distinct names that refer to the same memory cell. 
It is typically considered dangerous practice for programming languages, as more responsibility is put on the programmer to maintain memory address references.


\subsection{Readability}
Readability is one of the most important criteria for the language itself, as it is intended for beginners in programming. 
Therefore, the language must be set up, and only the most valuable data types have been used to set it up. It is also important for the language itself that the syntax setup is well thought out to create minor confusion for beginners.
This is done by keeping the orthogonality low, and the special words will be turned into reserved words to increase readability. 
Setting specific words as reserved words helps define particular designs in the program, allowing the language to be more readable.

\subsection{Writability}

The writability of the language is also an essential part of making it easy for the beginner to prepare programs with reduced problems. 
This is prevented by the high simplicity of the setup of the language and the simplification of the data types, which helps to make the syntax simpler to write down.
Also, the expressiveness of the language itself will be set up to lie between visual programming and a high-level language.

\subsection{Reliability}
The language's reliability deals with how the language works under different circumstances and how well it works. 
As the table shows, all the parameters help to influence the reliability of the language itself.
