\subsection{Statement Level Control Structures}
Users can change execution flow using control structures. These are if-statements and repeat-statements.
\subsubsection{If statements}
If statements are selection statements. These provide the means of choosing between two or more execution paths in a program\cite{sebesta_concepts_2016}.

In Dazel, if-statements are followed by an expression and then curly braces, which open a new scope. The statements within this scope are executed if the expression holds true.

\begin{lstlisting}[caption={Example of an if statement}, label={lst:ifexpression},escapechar=|]
if <expression>|\label{lst:exp}|
{
    Code...
}
\end{lstlisting}

In \ref{lst:ifexpression} on line \ref{lst:exp},  \verb|<expression>| denotes a placeholder expression.

\subsubsection{Repeat loops}
Repeat loops are iterative statements.

\begin{lstlisting}[caption={Example of loops}, label={lst:loops},escapechar=|]
repeat until <var> == 10|\label{lst:repeatuntil}|
{
    Code...
}

repeat forever|\label{lst:repeatforever}|
{
    Code...
}
\end{lstlisting}\fxfatal{Specify order of evaluation - once we've figured it out. See feedback 17/03}

In \ref{lst:loops} on line \ref{lst:repeatuntil}, \verb|<var>| denotes an arbitrary variable. It is expected that the expression will evaluate to true, otherwise the loop does the same as the \verb|repeat forever| loop on line \ref{lst:repeatforever}.

The \verb|repeat forever| loop repeats forever - or until the running instance terminates.
The \verb|repeat until| loop repeatedly executes the code in the scope immediately below it until the given expression evaluates to \verb|true|.