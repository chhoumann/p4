\subsection{Built-in functionality}
These are also keywords, but they represent library-like functions that a user can use to manipulate data. 
Note that positions are arrays consisting of two elements where the first element is the x coordinate and the second element is the y coordinate.
Explanations of these can be seen in \tabref{table:explanations_of_functions}.

\begin{longtable}{l|l}
    \multicolumn{1}{c|}{\textbf{\textit{Function}}}                                                                                                                                                                                                                                                  & \multicolumn{1}{c}{\textbf{\textit{Parameters}}}                                                                                                                                                                                                                                                                           \endfirsthead 
    \hline
    \begin{tabular}[c]{@{}l@{}}\textbf{Size(x, y):~}Sets the size of the map.\\\textbf{Return type:} void\\\textbf{Example:} Size(30, 24)\end{tabular}                                                                                                                                               & \begin{tabular}[c]{@{}l@{}}\textbf{x (int):} The width of the map in tiles.\\\textbf{y (int):~}The height of the map in tiles.\end{tabular}                                                                                                                                                                                \\ 
    \hline
    \begin{tabular}[c]{@{}l@{}}\textbf{Walls(tile):} Sets the outermost\\tiles on each side of the map to \\a certain tile type.\\\textbf{Return type:} void\\\textbf{Example:} Walls("Stone.png")\end{tabular}                                                                                      & \begin{tabular}[c]{@{}l@{}}\textbf{tile (string):} The name of the image file\\for the tile.\end{tabular}                                                                                                                                                                                                                  \\ 
    \hline
    \begin{tabular}[c]{@{}l@{}}\textbf{Floor(tile):\textit{~}}Sets the floor layer of each\\tile to be of a certain tile type.\\\textit{}\textbf{Return type:} void\\\textbf{Example:~}Floor("Grass.png")\end{tabular}                                                                               & \begin{tabular}[c]{@{}l@{}}\textbf{tile (string):} The name of the image file \\for the tile.\end{tabular}                                                                                                                                                                                                                 \\ 
    \hline
    \begin{tabular}[c]{@{}l@{}}\textbf{Line(pos1, pos2, tile):} Creates a line \\of a certain tile type.\\Takes two coordinates denoting start\\and end, as well as the desired tile type\textit{.}\\\textbf{Return type:} void\\\textbf{Example:} Line([2, 2], [2, 4], "Stone.png")\end{tabular}      & \begin{tabular}[c]{@{}l@{}}\textbf{pos1 (\textbf{Array}), pos2 (\textbf{Array}):} An array with\\length 2 where the first element is the \\x position and the second element is the \\y position.\\\textbf{tile(string):} The name of the image file\\for the tile.\end{tabular}                                           \\ 
    \hline
    \begin{tabular}[c]{@{}l@{}}\textbf{Square(pos, size, tile):~}Creates a \\square of a certain tile type.\\Takes a coordinate for the center \\position, a size, and a tile type.\\\textbf{Return type:} void\\\textbf{Example:} Square([8, 8], 4, "Stone.png")\end{tabular}                       & \begin{tabular}[c]{@{}l@{}}\textbf{pos (\textbf{Array}):~}An array with length 2\\where the first element is the x position\\and the second element is the y position.\\\textbf{size (int):} An integer denoting the size of\\the square.\\\textbf{tile (string):} The name of the image file\\for the tile.\end{tabular}  \\ 
    \hline
    \begin{tabular}[c]{@{}l@{}}\textbf{MoveLeft()}\\\textbf{Return type:} void\end{tabular}                                                                                                                                                                                                          & \begin{tabular}[c]{@{}l@{}}Moves an entity one tile left.\\Used when specifying move patterns.\end{tabular}                                                                                                                                                                                                                \\ 
    \hline
    \begin{tabular}[c]{@{}l@{}}\textbf{MoveUp()}\\\textbf{Return type:} void\end{tabular}                                                                                                                                                                                                            & \begin{tabular}[c]{@{}l@{}}Moves an entity one tile up.\\Used when specifying move patterns.\end{tabular}                                                                                                                                                                                                                  \\ 
    \hline
    \begin{tabular}[c]{@{}l@{}}\textbf{MoveDown()}\\\textbf{Return type:} void\end{tabular}                                                                                                                                                                                                          & \begin{tabular}[c]{@{}l@{}}Moves an entity one tile down.\\Used when specifying move patterns.\end{tabular}                                                                                                                                                                                                                \\ 
    \hline
    \begin{tabular}[c]{@{}l@{}}\textbf{MoveUp()}\\\textbf{Return type:}\textit{ void}\end{tabular}                                                                                                                                                                                                   & \begin{tabular}[c]{@{}l@{}}Moves an entity one tile up.\\Used when specifying move patterns.\end{tabular}                                                                                                                                                                                                \\ 
    \hline
    \begin{tabular}[c]{@{}l@{}}\textbf{PlayerInAggroRange()}\\\textbf{Return type:} bool\end{tabular}                                                                                                                                                                                                & \begin{tabular}[c]{@{}l@{}}Returns a boolean indicating whether \\the player is within the\textasciitilde{}aggro range of a \\given entity.\\Can only be called within the scope of\\an entity.\end{tabular}                                                                                                               \\ 
    \hline
    \begin{tabular}[c]{@{}l@{}}\textbf{Exit(pos, screen): }Creates an exit at\\the given position which takes the\\player to the specified screen when\\interacted with.\\\textbf{\textbf{Return type:}}~\textit{Exit}\end{tabular}                                                                  & \begin{tabular}[c]{@{}l@{}}\textbf{pos (\textbf{Array}):}~An array with length 2\\where the first element is the x position\\and the second element is the y position.\\\textbf{screen~\textbf{\textbf{\textbf{(Screen):~}}}}The identifier of the \\screen to connect the exit to.\end{tabular}                           \\ 
    \hline
    \begin{tabular}[c]{@{}l@{}}\textbf{ScreenExit(dir, screen):} Creates a\\screen exit in a given direction such\\that the player is taken to a different\\screen when walking off-screen in the\\specified direction.\\\textbf{\textbf{\textbf{\textbf{Return type:}}}}~\textit{Exit}\end{tabular} & \begin{tabular}[c]{@{}l@{}}\textbf{dir (Direction): }The direction the player \\must walk off-screen to trigger the exit.\\\textbf{\textbf{screen (Screen):}}~The identifier of the \\screen to connect the exit to.\end{tabular}                                                                                          \\ 
    \hline
    \begin{tabular}[c]{@{}l@{}}\textbf{SpawnEntity(entity, pos):} Spawns the\\given entity at the given position.\\\textbf{Return type:} \textit{Entity}\end{tabular}                                                                                                                                & \begin{tabular}[c]{@{}l@{}}\textbf{\textbf{entity (Entity): }}The identifier of the entity to \\spawn.\\\textbf{pos (Array):}~An array with length 2\\where the first element is the x position\\and the second element is the y position.\end{tabular}                                                                    \\                                                                   
    \caption{Explanations of built-in functionalities}
    \label{table:explanations_of_functions}
\end{longtable}