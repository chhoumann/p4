\subsection{Variables}

Variables in programming, also called a program variable, is a way of creating an abstraction of one or more memory cells. In the book "Concepts of Programming Languages" by
Robert W. Sebesta\cite{sebesta_concepts_2016}, he describes that a variable can be characterized as a sextuple consisting of the attributes:

\begin{itemize}
    \item Name
    \item Address
    \item Type
    \item Value
    \item Lifetime
    \item Scope
\end{itemize}

In the following sections we will describe the basic theory behind each of the items, in the list above, and how we have chosen to approach them for our language design for Dazel.
As a note, we will not be concerning ourselves with address and lifetime. This is due to the fact, that they are not important since we will be designing an interpreter that makes use of the .NET runtime
and therefore will be out of our control. \fxfatal{Skal rettes - er ikke korrekt}
Also the discussion about value is a given, as it relates to the value stored in memory\cite{sebesta_concepts_2016}.


\subsubsection{Name}

Variable names in a programming language is a string of characters used to identify an entity in the program\cite{sebesta_concepts_2016}. A name can be defined to be consisting of either capital letters,
lower case letters, some symbols or a mixture of all.

Specifically for our language design we have decided that a variable name can be constructed with the following rules:

\begin{itemize}
    \item A name can have capital letters
    \item A name can have lower case letters
    \item A name can start with a capital letter, followed by lower case letters
    \item A name can start with lower case letter(s) and have a single or more letters be capital following the first letter
    \item A name can have an arbitrary mixture of capital and lower case letters
    \item A name can end with an underscore symbol, but not start with an underscore symbol
    \item The is no specified limit on the length of the name
\end{itemize}

The list of rules mentioned can be simplified using the regular expression: [a-zA-Z][a-zA-Z\_0-9]*. Any construction of a variable outside of these rules will be considered
a non-valid construction.

\subsection{Type}

The type of a variable is either an implicit or explicit declaration that determines the value range and the set of operations available to this declaration.

In our programming language, we decided to have four different types with corresponding array-types. These will be implicity declared.

\begin{multicols}{3}
    \begin{enumerate}
        \item Numbers\label{item:numbertype}
        \item Booleans\label{item:booleantype}
        \item Strings\label{item:stringtype}
        \item Tiles\label{item:tiletype}
        \item Arrays*\label{item:arraytype}
    \end{enumerate}
\end{multicols}

\ref{item:numbertype} should denote both integer and floating point types. The actual type will be determined through context. \ref{item:booleantype} will be used for conditional expressions and control flow.
\ref{item:stringtype} will be used for naming.
\ref{item:tiletype} is an important type: it is the type behind every tile in the generated games.
\ref{item:arraytype} means that every type in the list will have a corresponding array type. This is to say that arrays are parametric - for example, an array could be an array of numbers.

\subsection{Scope}
In this section we describe different types of scope rules, specifically static scopes, block scopes and dynamic scopes.
We will ultimately use this to determine the best fit for \dazel{}.

\subsubsection{Static Scopes}
Originally introduced by ALGOL 60, \texttt{static scoping} is the act of binding names to nonlocal variables such that the scope of a variable
can be determined prior to program execution. This way, one can determine the type of every variable via the source code.
Static-scoped languages can be split into two categories: those that allow sub-programs to be nested and those that do not.
The former creates nested static scopes while the latter only creates nested scopes through nested class definitions and blocks.
Static scope rules are quite commonly used in modern languages due to their emphasis on static type checking.\cite{sebesta_concepts_2016}

\subsection{Dynamic Scopes}
In contrast to static scoping, variables declared in languages with \texttt{dynamic scoping} are determined at runtime through the calling sequence of subprograms.

\begin{lstlisting}[caption={Example of dynamic scoping}, label={lst:DynamicScopeExample}]
function main() {
	function nested1() {
		var x = 7;
	}
	
	function nested2() {
		var y = x;
	}

	var x = 3;
}
\end{lstlisting}

In listing \ref{DynamicScopeExample}, assuming that dynamic scoping rules apply, the meaning of the identifier \texttt{x} cannot be determined at compile time.
Specifically, the declaration of an identifier is first searched for in the local function.
If no declaration for \texttt{x} is found, the search continues to the function's caller, its \texttt{dynamic parent}, and so on up the call-stack.\cite{sebesta_concepts_2016}

\subsubsection{Blocks}
The powerful concept of \texttt{block scoping} allows specific sections of code called blocks to have their own localized variables.
Variables declared within a block are local, meaning they cannot be accessed from outside the scope. However, nested blocks can still access variables from outer scopes.
A variable declaed within a block is allocated when the block is entered and deallocated when it is exited.\cite{sebesta_concepts_2016}

\subsubsection{Choice of Scope}
Due to the nature of action maps in games, we decided to use block scopes in order to separate the elements found within a screen.
In particular, a screen on the map is defined by a block containing nested blocks that describe the maps layout, its exits and the entities that spawn within the given screen.
For instance, the tilemap layout of a screen would be defined in a \texttt{Map} block containing definitions for tile placements.
This separation of screen elements made sense as it achieves decoupling between the different parts of a screen by default.
