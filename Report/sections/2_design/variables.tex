\subsection{Variables}

Variables in programming, also called a program variable, is a way of creating an abstraction of one or more memory cells.
In \cite{sebesta_concepts_2016}, a variable is characterized as a sextuple consisting of the attributes:

\begin{itemize}
    \item Name
    \item Address
    \item Type
    \item Value
    \item Lifetime
    \item Scope
\end{itemize}

In the following sections we will describe the basic theory behind each of the items, in the list above, and how we have chosen to approach them for our language design for \dazel{}.

\subsubsection*{Address, lifetime and values of variables}
Since \dazel{} will be translated to C\#, the addressing and lifetime of variables is primarily handled by the .NET runtime,
however we can influence it through the scope rules described in \secref{sec:semantics}.
In doing so, we can ensure that no two variables share the same address as long as they are declared within different scopes.
It is also worth noting that, while our language dictates the actual value of a variable, its memory allocation is still managed by the .NET runtime.

\subsubsection*{Name}

Variable names in a programming language is a string of characters used to identify an entity in the program\cite{sebesta_concepts_2016}. A name can be defined to be consisting of either capital letters,
lower case letters, some symbols or a mixture of all.

Specifically for our language design we have decided that a variable name can be constructed with the following rules:

\begin{itemize}
    \item A name can have capital letters
    \item A name can have lower case letters
    \item A name can start with a capital letter, followed by lower case letters
    \item A name can start with lower case letter(s) and have a single or more letters be capital following the first letter
    \item A name can have an arbitrary mixture of capital and lower case letters
    \item A name can end with an underscore symbol, but not start with an underscore symbol
    \item The is no specified limit on the length of the name
\end{itemize}

The list of rules mentioned can be simplified using the regular expression: [a-zA-Z][a-zA-Z\_0-9]*. Any construction of a variable outside of these rules will be considered
a non-valid construction. Note that, since keywords are reserved, variables cannot be named the same as any of the keywords defined in \secref{sec:keywords}.

How variables are defined in relation to their scope and type will be discussed in the description of the language semantics in \secref{sec:semantics}.

\subsubsection*{Member access}
In order to access properties within other objects, we have decided to include a member access functionality similar to what might be seen in a
object oriented language. Member access in \dazel{} provides the ability to access all user defined game objects, including screens, entities and move patterns as well
as built-in objects such as the player. To access a game object, one simply writes its identifier. To then access a property of the game object, one then writes
a dot, followed by the identifier of the desired property.

This is illustrated in code \snipref{lst:SampleScreen1} on line 47. In this example, the exit property with the identifier \texttt{Exit1} of \texttt{SampleScreen2} is accessed.
The effect of this is that the property in question becomes the value passed into the actual parameter of the \texttt{Exit} function call.

