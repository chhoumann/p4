\section{Structural Operational Semantics}
In the following, we will describe how one can use a structural operational semantics to define a transitions system for a language - that is, we are only concerned with the behavior of a program.
We will then use this to give a formal definition of the semantics for \dazel{}.

\subsection{Big- and small-step semantics}
Big- and small-step semantics are two different kinds of operational semantics used describe computations.
Given a single transition $\gamma \rightarrow \gamma' $, big-step semantics describes the entire computation where $\gamma$ is the start configuration and $\gamma'$ must be a terminal configuration. 
In contrast, a small semantics describes only a single step in a larger computation in which $\gamma'$ does not need to be a terminal configuration. 
\Figref{SSS_and_BSS} illustrates the different between the two types of operational semantics\cite{huttelTransitionsTreesIntroduction2010}.

\fig{0.75}{SSS_and_BSS}{The difference between big- and small-step semantics illustrated\cite{huttelTransitionsTreesIntroduction2010}.}

We have opted to give a big-step semantics for \dazel{} as the simplicity of the language does not facilitate the need to describe every individual computation in detail.  
\subsubsection*{Abstract Syntax}
An abstract syntax is a description of a programs' structure that is used to represent its behavior. 
The abstract syntax of a given language contains a collection of syntactic categories. 
For each of these, we give a finite set of formation rules that define how each item in the category can be constructed.
\Figref{AbstractSyntax} shows the abstract syntax \dazel{}.





\begin{figure}[htbp]
	\centering
	\textit{Syntatic categories}
	\vspace{4mm}


	\begin{tabular}{l l}
		$a \in \textbf{ArithmExp}$ & $\text{Arithmetic expressions}$ \\ 
		$n \in \textbf{Num}$ & $\text{Numerals}$ \\
		$x \in \textbf{Var}$ & $\text{Variables}$ \\
		$A_v \in \textbf{AssignV}$ & $\text{Variable assignments}$ \\
		$D_v \in \textbf{DeclV}$ & $\text{Variable declarations}$ \\
		$C_m \in \textbf{ContDecl}$ & $\text{Content declarations}$ \\
		$S \in \textbf{Stm}$ & $\text{Statements}$
	\end{tabular}

	\vspace{4mm}
	\textit{Formation rules}
	\begin{align*}
		S&::=x=a\mid \texttt{skip}\mid S_1;S_2\mid \texttt{repeat }\{S\}; \\
		a&::=n\mid x\mid a_1+a_2\mid a_1-a_2\mid a_1*a_2\mid a_1/a_2\mid (a_1) \\
		%VList&::=[a]\mid [a_1, \ldots, a_n]\mid [\epsilon] \\
		GOName&::= \textit{Screen} \mid \textit{Entity} \mid \textit{MovePattern} \\
		GOCont&::= \textit{Map} \mid \textit{Exits} \mid \textit{Entities} \mid \textit{OnScreenEntered} \mid \textit{Data}
	\end{align*}

	\caption{Abstract syntax for \dazel{}.}
	\label{fig:AbstractSyntax}
\end{figure}

\subsubsection*{Transition System}

\begin{figure}[htbp]
	\centering
	\begin{gather*}
		[PLUS_\text{BSS}] 
		\qquad \dfrac
		{s \vdash a_1 \rightarrow_a v_1 \; s \vdash a_2 \rightarrow_a v_2}
		{s \vdash a_1 + a_2 \rightarrow_a v}
		\qquad \text{ where }v = v_1 + v_2
		\\
		% 
		\\
		[MINUS_\text{BSS}] 
		\qquad \dfrac
		{s \vdash a_1 \rightarrow_a v_1 \; s \vdash a_2 \rightarrow_a v_2}
		{s \vdash a_1 - a_2 \rightarrow_a v}
		\qquad \text{ where }v = v_1 - v_2
		\\
		% 
		\\
		[MULTIPLY_\text{BSS}] 
		\qquad \dfrac
		{s \vdash a_1 \rightarrow_a v_1 \; s \vdash a_2 \rightarrow_a v_2}
		{s \vdash a_1 * a_2 \rightarrow_a v}
		\qquad \text{ where }v = v_1 \cdot v_2
		\\
		% 
		\\
		[DIVIDE_\text{BSS}] 
		\qquad \dfrac
		{s \vdash a_1 \rightarrow_a v_1 \; s \vdash a_2 \rightarrow_a v_2}
		{s \vdash a_1 / a_2 \rightarrow_a v}
		\qquad \text{ where }v = \dfrac{v_1}{v_2}
		\\
		% 
		\\
		[PARENTHESES_\text{BSS}] 
		\qquad \dfrac
		{s \vdash a_1 \rightarrow_a v_1}
		{s \vdash (a_1) \rightarrow_a v_1}
		\\
		% 
		\\
		[NUM_\text{BSS}] 
		\qquad s \vdash n \rightarrow_a v
		\qquad \text{ if } \mathbb{R}[\![n]\!] = v \quad \text{Where } \mathcal{R}:\textbf{Num}\rightarrow \mathbb{R}
		\\
		% 
		\\
		[VAR_\text{BSS}] 
		\qquad s \vdash x \rightarrow_a v
		\qquad \text{ if } sx = v
	\end{gather*}
	\caption{Big-step transition rules for \textbf{ArithmExp}.}
	\label{fig:BssArithm}
\end{figure}

\begin{figure}[htbp]
	\centering
	\begin{gather*}
		[ASSIGNMENT_\text{BSS}] 
		\qquad \langle x=a,s\rangle \rightarrow s[x\mapsto v]
		\qquad \text{ where }s\vdash a \rightarrow_a v
		\\
		% 
		\\
		[COMPOUND_\text{BSS}] 
		\qquad \dfrac
		{\langle S_1, s\rangle \rightarrow s'' \;\langle S_2, s''\rangle \rightarrow s'}
		{\langle S_1;S_2,s\rangle\rightarrow s'}
		\\
		% 
		\\
		[REPEAT_\text{BSS}] 
		\qquad \dfrac
		{\langle S_1, s\rangle \rightarrow s'' \;\langle \texttt{repeat} \{S\}, s\rangle \rightarrow s'}
		{\langle \texttt{repeat} \{S\}, s\rangle\rightarrow s'}
	\end{gather*}
	\caption{Big-step transition rules for \textbf{Stm}.}
	\label{fig:BssStm}
\end{figure}

\begin{figure}[htbp]
	\centering
	\begin{gather*}
		[GAMEOBJECT_\text{BSS}] 
		\qquad \dfrac
		{\langle S, s\rangle \rightarrow s'' \quad\langle GOName \{GOCont\}, s''\rangle \rightarrow s'}
		{\langle GOName \{GOCont\}, s\rangle\rightarrow s'}
		\\
		% 
		\\
		[CONTENT_\text{BSS}] 
		\qquad \dfrac
		{\langle S, s\rangle \rightarrow s'' \quad\langle GOCont \{S\}, s''\rangle \rightarrow s'}
		{\langle GOCont \{S\}, s\rangle\rightarrow s'}
		\\
		% 
		\\
	\end{gather*}
	\caption{Big-step transition rules for \textbf{ContDecl}.}
	\label{fig:BssStm}
\end{figure}


\subsubsection*{Transition System}

For our operational semantics we have a transition system defined for each expression and declaration. 
The structure for the transition system will be a triple $(\Gamma,\rightarrow,T)$ where $\Gamma$ is a set of configurations, $\rightarrow$ is the transition relation, which is a subset of $\Gamma\times\Gamma$, and $T\subseteq \Gamma$ is a set of terminal configurations. \cite{huttelTransitionsTreesIntroduction2010}


\begin{gather*} 
	\{ \Gamma_\text{AExp}, \rightarrow_a, T_\text{Aexp} \} \\ 
	\textit{Configurations: } \Gamma_\text{Aexp} = \text{Aexp} \cup \mathbb{R} \\ 
	\textit{Terminal configurations: } T_\text{Aexp} = \mathbb{R}
\end{gather*} 

\begin{gather*} 
	\{ \Gamma_\text{BExp}, \rightarrow_a, T_\text{Bexp} \} \\ 
	\textit{Configurations: } \Gamma_\text{Bexp} = \text{Bexp} \cup \{tt, f\!f\} \\ 
	\textit{Terminal configurations: } T_\text{Bexp} = \{tt, f\!f\}
\end{gather*}

\begin{gather*} 
	\{ \Gamma_\text{AExp}, \rightarrow_a, T_\text{Aexp} \} \\ 
	\textit{Configurations: } \Gamma_\text{Aexp} = \text{Aexp} \cup \mathbb{R} \\ 
	\textit{Terminal configurations: } T_\text{Aexp} = \mathbb{R}
\end{gather*} 

\subsubsection*{Syntactic Categories}
\subsubsection*{Formation Rules}
\subsubsection*{Big Step Semantic Rules For \dazel{}}