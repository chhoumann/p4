\subsection{Structural Operational Semantics}

\subsubsection*{Abstract Syntax}
An abstract syntax is a description of a programs' structure that is used to represent its behavior. 
The abstract syntax of a given language contains a collection of syntactic categories. 
For each of these, we give a finite set of formation rules that define how each item in the category can be constructed.
\Figref{AbstractSyntax} shows the abstract syntax \dazel{}.

\begin{figure}[htbp]
	\centering
	\textit{Syntatic categories}
	\vspace{4mm}


	\begin{tabular}{l l}
		$a \in \textbf{ArithmExp}$ & $\text{Arithmetic expressions}$ \\ 
		$n \in \textbf{Num}$ & $\text{Numerals}$ \\
		$x \in \textbf{Var}$ & $\text{Variables}$ \\
		$A_v \in \textbf{AssignV}$ & $\text{Variable assignments}$ \\
		$D_v \in \textbf{DeclV}$ & $\text{Variable declarations}$ \\
		$C_m \in \textbf{ContDecl}$ & $\text{Content declarations}$ \\
		$S \in \textbf{Stm}$ & $\text{Statements}$
	\end{tabular}

	\vspace{4mm}
	\textit{Formation rules}
	\begin{align*}
		S&::=x=a\mid \texttt{skip}\mid S_1;S_2\mid \texttt{repeat }\{S\}; \\
		a&::=n\mid x\mid a_1+a_2\mid a_1-a_2\mid a_1*a_2\mid a_1/a_2\mid (a_1) \\
		VList&::=[a]\mid [a_1, \ldots, a_n]\mid [\epsilon]
	\end{align*}

	\caption{Abstract syntax for \dazel{}.}
	\label{fig:AbstractSyntax}
\end{figure}

\begin{figure}[htbp]
	\centering
	\begin{gather*}
		[PLUS_\text{BSS}] 
		\qquad \dfrac
		{s \vdash a_1 \rightarrow_a v_1 \; s \vdash a_2 \rightarrow_a v_2}
		{s \vdash a_1 + a_2 \rightarrow_a v}
		\qquad \text{ where }v = v_1 + v_2
		\\
		% 
		\\
		[MINUS_\text{BSS}] 
		\qquad \dfrac
		{s \vdash a_1 \rightarrow_a v_1 \; s \vdash a_2 \rightarrow_a v_2}
		{s \vdash a_1 - a_2 \rightarrow_a v}
		\qquad \text{ where }v = v_1 - v_2
		\\
		% 
		\\
		[MULTIPLY_\text{BSS}] 
		\qquad \dfrac
		{s \vdash a_1 \rightarrow_a v_1 \; s \vdash a_2 \rightarrow_a v_2}
		{s \vdash a_1 * a_2 \rightarrow_a v}
		\qquad \text{ where }v = v_1 \cdot v_2
		\\
		% 
		\\
		[DIVIDE_\text{BSS}] 
		\qquad \dfrac
		{s \vdash a_1 \rightarrow_a v_1 \; s \vdash a_2 \rightarrow_a v_2}
		{s \vdash a_1 / a_2 \rightarrow_a v}
		\qquad \text{ where }v = \dfrac{v_1}{v_2}
		\\
		% 
		\\
		[PARENTHESES_\text{BSS}] 
		\qquad \dfrac
		{s \vdash a_1 \rightarrow_a v_1}
		{s \vdash (a_1) \rightarrow_a v_1}
		\\
		% 
		\\
		[NUM_\text{BSS}] 
		\qquad s \vdash n \rightarrow_a v
		\qquad \text{ if } \mathbb{R}[\![n]\!] = v
		\\
		% 
		\\
		[VAR_\text{BSS}] 
		\qquad s \vdash x \rightarrow_a v
		\qquad \text{ if } sx = v
	\end{gather*}
	\caption{Big-step transition rules for \textbf{ArithmExp}.}
\end{figure}

\begin{figure}[htbp]
	\centering
	\begin{gather*}
		[ASSIGNMENT_\text{BSS}] 
		\qquad \langle x=a,s\rangle \rightarrow s[x\mapsto v]
		\qquad \text{ where }s\vdash a \rightarrow_a v
		\\
		% 
		\\
		[COMPOUND_\text{BSS}] 
		\qquad \dfrac
		{\langle S_1, s\rangle \rightarrow s'' \;\langle S_2, s''\rangle \rightarrow s'}
		{\langle S_1;S_2,s\rangle\rightarrow s'}
		\\
		% 
		\\
		[REPEAT_\text{BSS}] 
		\qquad \dfrac
		{\langle S_1, s\rangle \rightarrow s'' \;\langle \texttt{repeat} \{S\}, s\rangle \rightarrow s'}
		{\langle \texttt{repeat} \{S\}, s\rangle\rightarrow s'}
	\end{gather*}
	\caption{Big-step transition rules for \textbf{Stm}.}
\end{figure}


\subsubsection*{Transition System}