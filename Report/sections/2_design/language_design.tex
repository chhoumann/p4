\chapter{Language design}
This chapter details the design of \dazel.
The language is intended to be a transitional language from visual programming to high-level language. Based on the problem analysis, it has been possible to specify the design for the user.
The target group is beginners in programming who are younger people. Therefore a programming language to make computer games has been built. This has been chosen as young people can relate more to video games as their hobby. \fxfatal{Kunne godt uddybes en del:
    "This has been chosen as young people can relate more to video games as their hobby" - skrev vi ikke lidt om det her i starten? måske lidt at hente derfra.
    Men i hvert fald en kilde på at understøtte påstanden. - John's review on PR}
This will help them to understand the principles of how a game should work more easily, and therefore be able to start programming. \fxfatal{Maybe give an example of what principles? - John's review on PR}
It was decided to develop a language inspired by the game "the Legend of Zelda: Link’s Awakening" into the programming language, as it is a familiar game with basic game terminologies.
This helps to set a framework for what the program language should be capable of.
In addition, "Unity" has been used as a program to test different functions to find out what could work well for the program language.

\section{Code examples}

\begin{lstlisting}[caption={Example screen}, label={lst:SampleScreen1},escapechar=|]
// ./Screens/SampleScreen1.txt
Screen SampleScreen1 
{
    Map 
    {
        Size(30, 24);

        Walls(Stone); 
        Floor(Grass);

        Line(TopLeft, TopRight, Cliff);
        Line([2, 2], [2, 4], Cliff);

        Square(position, size, Cliff);
        Square([8, 8], 4, Cliff);
    }

    OnScreenEntered
    {
        if Player.CompletedQuest(...) 
        {
            SetTile([4, 0], Stair);
        }
    }

    Entities
    {
        SpawnEntity(Skeleton1, [4, 4]).SetMovePattern(Square1);
    }
    
    Exits 
    {
        // Two different ways to create an exit
        // Specific tile(s) or whole screen side
        Exit1 = Exit([4, 0], SampleScreen2.Exit1);
        ScreenExit(Bottom, SampleScreen2);
    }
}
\end{lstlisting}

Code snippet~\ref{lst:SampleScreen1} shows an example of a screen. A screen represents a single area within the game, and each of the functions used within the Screen
define the layout, what happens as the screen is loaded, the entities that should be present as well as the exits of the level.
Objects such as Skeleton1 and the movement pattern Square1 on line 28 will be defined in a separate file.
The purpose of defining these separetely is to allow ease of reuseability and have a clear separation of concerns.

\section{Language design}

To start designing the programming language, we set out to specify a minimum viable product (MVP). We decided to start small but with expansion in mind.

\subsection{Keywords}\label{sec:keywords}
\Tabref{table:KeyWordsTable} introduces the reserved keywords in \dazel{}.  A convention has been introduced that defines all block-indicators to be denoted by an upper-case first character.

\begin{table}[h!]
    \centering
    \begin{tabular}{l|l}
    \textbf{Keyword}             & \textbf{Description}                                                                                                                                                                                                                                            \\ 
    \hline
    Player                       & \begin{tabular}[c]{@{}l@{}}A globally accessible reference\\ to the player. This allows the\\ programmer to create interactions \\involving the player's properties such \\as health or position.\end{tabular}  \\ 
    \hline
    Screen                       & \begin{tabular}[c]{@{}l@{}}Denotes a screen game object, which is a \\container for other blocks such as \\'Map', 'OnScreenEntered', 'Entities', \\and 'Exits'.\end{tabular}                                                                                          \\ 
    \hline
    Entity                       & \begin{tabular}[c]{@{}l@{}}Denotes an entity game object used to define the \\behavior of an entity.\end{tabular}                                                                                                                                                     \\ 
    \hline
    MovePattern                  & \begin{tabular}[c]{@{}l@{}}Denotes a move pattern game object which \\can be used to apply movement to\\entities.\end{tabular}                                                                                                                                        \\ 
    \hline
    Exits, ScreenExit, FloorExit & \begin{tabular}[c]{@{}l@{}}Denotes exits or transitions to other screens.\\These are usually triggered by player \\movement.\end{tabular}                                                                                                                       \\ 
    \hline
    Map                          & \begin{tabular}[c]{@{}l@{}}Denotes a map block which can be used\\to define a screen's size and its tile layout.\end{tabular}                                                                                                                                   \\ 
    \hline
    OnUpdate                     & \begin{tabular}[c]{@{}l@{}}Denotes an update block whose code is\\executed every frame.\end{tabular}                                                                                                                                                            \\ 
    \hline
    Data                         & \begin{tabular}[c]{@{}l@{}}Denotes a data block which can define\\properties such as health on entities.\end{tabular}                                                                                                                                           \\ 
    \hline
    Pattern                      & \begin{tabular}[c]{@{}l@{}}Denotes a pattern block which can be used\\to define specific movement patterns\\that may be reused across\\multiple entities.\end{tabular}                                                                                          \\ 
    \hline
    Entities                     & \begin{tabular}[c]{@{}l@{}}Denotes an entities block which can be\\used to define placement of entities on\\a given screen.\end{tabular}                                                                                                                        \\
    \hline
    until                        & \begin{tabular}[c]{@{}l@{}}Used to indicate that there is a condition\\for a repeat loop.\end{tabular}                                                                                                                                                          \\ 
    \hline
    or                           & Logical OR operator.                                                                                                                                                                                                                                            \\ 
    \hline
    and                          & Logical AND operator.                                                                                                                                                                                                                                                
    \end{tabular}
    \caption{Table explaining reserved keywords in \dazel}
    \label{table:KeyWordsTable}
\end{table}

\subsection{Built-in functionality}
Since we do not allow function definitions, we provide the programmer with a set of built-in functions that may be used to create their game. Explanations of these can be seen in \tabref{table:explanations_of_functions}.
In the table, we use \textbf{pos} to denote positions, which are arrays consisting of two elements where the first element is the x coordinate and the second element is the y coordinate.

\begin{longtable}{l|l}
    \multicolumn{1}{c|}{\textbf{\textit{Function}}}                                                                                                                                                                                                                                                  & \multicolumn{1}{c}{\textbf{\textit{Parameters}}}                                                                                                                                                                                                                                                                           \endfirsthead 
    \hline
    \begin{tabular}[c]{@{}l@{}}\textbf{Size(x, y):~}Sets the size of the map.\\\textbf{Return type:} void\\\textbf{Example:} Size(30, 24)\end{tabular}                                                                                                                                               & \begin{tabular}[c]{@{}l@{}}\textbf{x (int):} The width of the map in tiles.\\\textbf{y (int):~}The height of the map in tiles.\end{tabular}                                                                                                                                                                                \\ 
    \hline
    \begin{tabular}[c]{@{}l@{}}\textbf{Walls(tile):} Sets the outermost\\tiles on each side of the map to \\a certain tile type.\\\textbf{Return type:} void\\\textbf{Example:} Walls("Stone.png")\end{tabular}                                                                                      & \begin{tabular}[c]{@{}l@{}}\textbf{tile (string):} The name of the image file\\for the tile.\end{tabular}                                                                                                                                                                                                                  \\ 
    \hline
    \begin{tabular}[c]{@{}l@{}}\textbf{Floor(tile):\textit{~}}Sets the floor layer of each\\tile to be of a certain tile type.\\\textit{}\textbf{Return type:} void\\\textbf{Example:~}Floor("Grass.png")\end{tabular}                                                                               & \begin{tabular}[c]{@{}l@{}}\textbf{tile (string):} The name of the image file \\for the tile.\end{tabular}                                                                                                                                                                                                                 \\ 
    \hline
    \begin{tabular}[c]{@{}l@{}}\textbf{Line(pos1, pos2, tile):} Creates a line \\of a certain tile type.\\Takes two coordinates denoting start\\and end, as well as the desired tile type\textit{.}\\\textbf{Return type:} void\\\textbf{Example:} Line([2, 2], [2, 4], "Stone.png")\end{tabular}      & \begin{tabular}[c]{@{}l@{}}\textbf{pos1 (\textbf{Array}), pos2 (\textbf{Array}):} The start and\\end positions of the line.\\\textbf{tile(string):} The name of the image file\\for the tile.\end{tabular}                                           \\ 
    \hline
    \begin{tabular}[c]{@{}l@{}}\textbf{Square(pos, size, tile):~}Creates a \\square of a certain tile type.\\Takes a coordinate for the center \\position, a size and a tile type.\\\textbf{Return type:} void\\\textbf{Example:} Square([8, 8], 4, "Stone.png")\end{tabular}                       & \begin{tabular}[c]{@{}l@{}}\textbf{pos (\textbf{Array}):~}The position of the center\\of the square.\\\textbf{size (int):} An integer denoting the size of\\the square.\\\textbf{tile (string):} The name of the image file\\for the tile.\end{tabular}  \\ 
    \hline
    \begin{tabular}[c]{@{}l@{}}\textbf{MoveLeft()}\\\textbf{Return type:} void\end{tabular}                                                                                                                                                                                                          & \begin{tabular}[c]{@{}l@{}}Moves an entity one tile left.\\Used when specifying move patterns.\end{tabular}                                                                                                                                                                                                                \\ 
    \hline
    \begin{tabular}[c]{@{}l@{}}\textbf{MoveUp()}\\\textbf{Return type:} void\end{tabular}                                                                                                                                                                                                            & \begin{tabular}[c]{@{}l@{}}Moves an entity one tile up.\\Used when specifying move patterns.\end{tabular}                                                                                                                                                                                                                  \\ 
    \hline
    \begin{tabular}[c]{@{}l@{}}\textbf{MoveDown()}\\\textbf{Return type:} void\end{tabular}                                                                                                                                                                                                          & \begin{tabular}[c]{@{}l@{}}Moves an entity one tile down.\\Used when specifying move patterns.\end{tabular}                                                                                                                                                                                                                \\ 
    \hline
    \begin{tabular}[c]{@{}l@{}}\textbf{MoveUp()}\\\textbf{Return type:}\textit{ void}\end{tabular}                                                                                                                                                                                                   & \begin{tabular}[c]{@{}l@{}}Moves an entity one tile up.\\Used when specifying move patterns.\end{tabular}                                                                                                                                                                                                \\ 
    \hline
    \begin{tabular}[c]{@{}l@{}}\textbf{PlayerInAggroRange()}\\\textbf{Return type:} bool\end{tabular}                                                                                                                                                                                                & \begin{tabular}[c]{@{}l@{}}Returns a boolean indicating whether \\the player is within the\textasciitilde{}aggro range of a \\given entity.\\Can only be called within the scope of\\an entity.\end{tabular}                                                                                                               \\ 
    \hline
    \begin{tabular}[c]{@{}l@{}}\textbf{Exit(pos, screen): }Creates an exit at\\the given position which takes the\\player to the specified screen when\\interacted with.\\\textbf{\textbf{Return type:}}~\textit{Exit}\end{tabular}                                                                  & \begin{tabular}[c]{@{}l@{}}\textbf{pos (\textbf{Array}):}~The position at which to\\place the exit.\\\textbf{screen~\textbf{\textbf{\textbf{(Screen):~}}}}The identifier of the \\screen to connect the exit to.\end{tabular}                           \\ 
    \hline
    \begin{tabular}[c]{@{}l@{}}\textbf{ScreenExit(dir, screen):} Creates a\\screen exit in a given direction such\\that the player is taken to a different\\screen when walking off-screen in the\\specified direction.\\\textbf{\textbf{\textbf{\textbf{Return type:}}}}~\textit{Exit}\end{tabular} & \begin{tabular}[c]{@{}l@{}}\textbf{dir (Direction): }The direction the player \\must walk off-screen to trigger the exit.\\\textbf{\textbf{screen (Screen):}}~The identifier of the \\screen to connect the exit to.\end{tabular}                                                                                          \\ 
    \hline
    \begin{tabular}[c]{@{}l@{}}\textbf{SpawnEntity(entity, pos):} Spawns the\\given entity at the given position.\\\textbf{Return type:} \textit{Entity}\end{tabular}                                                                                                                                & \begin{tabular}[c]{@{}l@{}}\textbf{\textbf{entity (Entity): }}The identifier of the entity to \\spawn.\\\textbf{pos (Array):}~The position at which to\\spawn the entity.\end{tabular}                                                                    \\                                                                   
    \caption{Explanations of built-in functionalities}
    \label{table:explanations_of_functions}
\end{longtable}

\subsection{Statement Level Control Structures}
Users can change execution flow using control structures. These are if-statements and repeat-statements.
\subsubsection{If statements}
If statements are selection statements. These provide the means of choosing between two or more execution paths in a program\cite{sebesta_concepts_2016}.

In Dazel, if-statements are followed by an expression and then curly braces, which open a new scope. The statements within this scope are executed if the expression holds true.

\begin{lstlisting}[caption={Example of an if statement}, label={lst:ifexpression},escapechar=|]
if <expression>|\label{lst:exp}|
{
    Code...
}
\end{lstlisting}

In \ref{lst:ifexpression} on line \ref{lst:exp},  \verb|<expression>| denotes a placeholder expression.

\subsubsection{Repeat loops}
Repeat loops are iterative statements.

\begin{lstlisting}[caption={Example of loops}, label={lst:loops},escapechar=|]
repeat until <var> == 10|\label{lst:repeatuntil}|
{
    Code...
}

repeat forever|\label{lst:repeatforever}|
{
    Code...
}
\end{lstlisting}\fxfatal{Specify order of evaluation - once we've figured it out. See feedback 17/03}

In \ref{lst:loops} on line \ref{lst:repeatuntil}, \verb|<var>| denotes an arbitrary variable. It is expected that the expression will evaluate to true, otherwise the loop does the same as the \verb|repeat forever| loop on line \ref{lst:repeatforever}.

The \verb|repeat forever| loop repeats forever - or until the running instance terminates.
The \verb|repeat until| loop repeatedly executes the code in the scope immediately below it until the given expression evaluates to \verb|true|.
\subsection{Operator symbols}
\fxfatal{Work in progress. Should this be here?}
See \tabref{table:operatorsymbols}.

\begin{table}[h!]
    \centering
    \begin{tabular}{ |c|c|c||c|c|c|c| }
        \hline
        \textbf{Category} & \textbf{Operator} & \textbf{Associativity} & \texttt{Number} & \texttt{String} & \texttt{Boolean} & \texttt{Tile} \\                              
        \hline

        \multirow{6}{*}{Relational} 
        & \texttt{<}
        & $\rightarrow$ 
        & Type1 
        & Type2 
        & Type3 
        & Type4 
        \\
        
         
        & \texttt{>} 
        & $\rightarrow$ 
        & Type1 
        & Type2 
        & Type3 
        & Type4 
        \\

        
        & \texttt{==} 
        & $\rightarrow$ 
        & Type1 
        & Type2 
        & Type3 
        & Type4 
        \\

         
        & \texttt{!=} 
        & $\rightarrow$ 
        & Type1 
        & Type2 
        & Type3 
        & Type4 
        \\

         
        & \texttt{>=} 
        & $\rightarrow$ 
        & Type1 
        & Type2 
        & Type3 
        & Type4 
        \\

         
        & \texttt{<=} 
        & $\rightarrow$ 
        & Type1 
        & Type2 
        & Type3 
        & Type4 
        \\\hline

        \multirow{2}{*}{Logic}
        & \texttt{and} 
        & $\rightarrow$ 
        & Type1 
        & Type2 
        & Type3 
        & Type4 
        \\
 
        & \texttt{or} 
        & $\rightarrow$ 
        & Type1 
        & Type2 
        & Type3 
        & Type4 
        \\\hline
        
        Assignment 
        & \texttt{=}
        & $\leftarrow$ 
        & Type1 
        & Type2 
        & Type3 
        & Type4 
        \\\hline
        
        \multirow{4}{*}{Arithmetic} 
        & \texttt{+}
        & $\rightarrow$ 
        & Type1 
        & Type2 
        & Type3 
        & Type4 
        \\
        
         
        & \texttt{-}
        & $\rightarrow$ 
        & Type1 
        & Type2 
        & Type3 
        & Type4 
        \\
         
        & \texttt{*} 
        & $\rightarrow$ 
        & Type1 
        & Type2 
        & Type3 
        & Type4 
        \\
         
        & \texttt{/}
        & $\rightarrow$ 
        & Type1 
        & Type2 
        & Type3 
        & Type4 
        \\\hline
    \end{tabular}
    \caption{Operation symbols}
    \label{table:operatorsymbols}
\end{table}

\fxfatal{Fix types. Depends on implementation.}
\subsection{Context-Free Grammar Derivations}
There are two types of derivations; leftmost- and rightmost-derivations\cite{sebesta_concepts_2016}.

A \textit{leftmost derivation} is where the leftmost nonterminal symbol is replaced at each step of the derivation.

A \textit{rightmost derivation} is where the rightmost nonterminal is replaced at each step of the derivation.

Typically, bottom-up parsing algorithms use rightmost derivations in reverse. These are the LR parser and its variations. Likewise, most top-down parsers use leftmost derivations - this being the LL parser and its variations.
\subsection{Methods}
\subsection{Variables}

Variables in programming, also called a program variable, is a way of creating an abstraction of one or more memory cells.
In \cite{sebesta_concepts_2016}, a variable is characterized as a sextuple consisting of the attributes:

\begin{itemize}
    \item Name
    \item Address
    \item Type
    \item Value
    \item Lifetime
    \item Scope
\end{itemize}

In the following sections we will describe the basic theory behind each of the items, in the list above, and how we have chosen to approach them for our language design for \dazel{}.

\subsubsection*{Address, lifetime and values of variables}
Since \dazel{} will be translated to C\#, the addressing and lifetime of variables is primarily handled by the .NET runtime,
however we can influence it through the scope rules described in \secref{sec:semantics}.
In doing so, we can ensure that no two variables share the same address as long as they are declared within different scopes.
It is also worth noting that, while our language dictates the actual value of a variable, its memory allocation is still managed by the .NET runtime.

\subsubsection*{Name}

Variable names in a programming language is a string of characters used to identify an entity in the program\cite{sebesta_concepts_2016}. A name can be defined to be consisting of either capital letters,
lower case letters, some symbols or a mixture of all.

Specifically for our language design we have decided that a variable name can be constructed with the following rules:

\begin{itemize}
    \item A name can have capital letters
    \item A name can have lower case letters
    \item A name can start with a capital letter, followed by lower case letters
    \item A name can start with lower case letter(s) and have a single or more letters be capital following the first letter
    \item A name can have an arbitrary mixture of capital and lower case letters
    \item A name can end with an underscore symbol, but not start with an underscore symbol
    \item The is no specified limit on the length of the name
\end{itemize}

The list of rules mentioned can be simplified using the regular expression: [a-zA-Z][a-zA-Z\_0-9]*. Any construction of a variable outside of these rules will be considered
a non-valid construction.

How variables are defined in relation to their scope and type will be discussed in the description of the language semantics in \secref{sec:semantics}.

\subsubsection*{Member access}
In order to access properties within other objects, we have decided to include a member access functionality similar to what might be seen in a
object oriented language. Member access in \dazel{} provides the ability to access all user defined game objects, including screens, entities and move patterns as well
as built-in objects such as the player. To access a game object, one simply writes its identifier. To then access a property of the game object, one then writes
a dot, followed by the identifier of the desired property.

This is illustrated in code \snipref{lst:SampleScreen1} on line 47. In this example, the exit property with the identifier \texttt{Exit1} of \texttt{SampleScreen2} is accessed.
The effect of this is that the property in question becomes the value passed into the actual parameter of the \texttt{Exit} function call.


\subsection{Expressions}
A major part of any programming language is expressions of which there are several types.
In the following, we will introduce these and discuss their relevance to \dazel{}.

\subsubsection*{Arithmetic expressions}
Arithmetic expressions are basic mathematical expressions. 
They consist of operators, operands, and function calls. 
Operators can either be unary, binary or ternary meaning they have either one, two or three operands.
Finally, binary operators are usually infix, meaning they appear between operands.

In \dazel{}, tiles on screens are placed on a grid with the bottom left corner being located at (0, 0) and only growing positively.
Because of this, the language is only concerned with positive numbers, and consequently it exclusively features a variety of binary operators. 

- arithmetic Expressions
- operator evaluation order
	- precedence
	- associativity
	- parantheses
	- side effects
- relational expressions
- boolean expressions

\subsection{Language Grammar} \label{langGram}

In this section we will describe how to formally specify a language using constructs such as Backus-Naur form (BNF), Extended Backus-Naur form(EBNF).
Following these two topics we will introduce the concept of derivation, and which type of derivation we have chosen for our language. 
Finally, we will give a justification for this choice as well as concrete examples of how all the previously mentioned elements are represented in our language.

\subsection{Backus-Naur form}

Backus-Naur form was a an evolution of a formal notation used to describe the language ALGOL 58, introduced by John Backus in 1959. The evolution happened as a result of modifications
made to this notation by Peter Naur in 1960, when he was describing ALGOL 60. Hence why the notation became known as Backus-Naur form (BNF)\cite{sebesta_concepts_2016}. 

As alluded to, BNF is used as a formal notation to describe the language, or more specifically its grammar. The construction of this grammar is made up of a collection of production rules that are formed as shown in the example (indsæt eksmpel).

In (forrige indsatte figur) there is a left hand side (LHS) and a right hand side (RHS), separated by an arrow. The LHS is a variable, also called the abstraction, that is being defined and the RHS is the definition.
This definition on the RHS consists of non-terminals and terminals. A non-terminal is a variable (sometimes called a symbol) that is an abstraction that have references to other abstractions. 
These non-terminal variables can have multiple distinct definitions each being a possible form of the syntax. If a non-terminal has multiple definitions these can either be represented as multiple 
rules as seen in fig.. or as a single rule where each definition is seperated by an OR operator: '|'. Terminals correlates to a piece of the output, as shown in figure (indsæt fig). 
Terminals are also referred to as lexemes and tokens\cite{sebesta_concepts_2016}. 

\subsection{Extended Backus-Naur form}

Due to some people experiencing inconveniences, BNF has seen several different extensions.
These extended versions are commonly referred to as Extended BNF or just EBNF. 
EBNF is not any more powerful than BNF, rather it simply adjusts and adds to the syntax in an attempt to make it more readable.
Whether EBNF succeeds in that regard is completely subjective. 

Most versions of ENBF usually include three extensions:
- The use of square brackets to denote and optional part of the right-hand side of a production
- The use of curly brackets to indicate that the enclosed part may be iterated any number of times
- The use of the OR operator, |, to denote multiple-choice options for a production

Despite there being a standardized version of EBNF, it is rarely ever used. 
Whichever version a language designer decides to use is up to preference.

\subsection{Derivation}


\subsection{Justification}


\subsection{Member Access}

\section{Semantics for Dazel}

In this section we will be presenting, in an informal manner, the semantics of \dazel. 
Firstly, we will present the type rules which is then followed by a presentation of the scope rules.



\subsection*{Type rules}
\dazel has implicitly declared strongly types.
This means the programmer does not need to explicitly define the type of every variable, however, once declared, a variable
cannot dynamically change type.
For example, as can be seen in \ref{lst:IllegalAssignment}, \textbf{Exit1} is declared to be of type \textbf{Exit} so it cannot be be
assigned to an integer afterwards.

\begin{lstlisting}[caption={Example of an illegal assingment}, label={lst:IllegalAssignment},escapechar=|]
// ./Screens/SampleScreen1.txt
Screen SampleScreen1 
{
	Exits 
	{
		Exit1 = Exit([4, 0], SampleScreen2.Exit1);
		Exit1 = 1; // Not legal because Exit1 has type Exit
	}
}
\end{lstlisting}

The decision to use this system is based on the fact that strongly typed languages offer greater safety, however being required to
specify the type of each variable as part of its declaration is often confusing to beginners as mentioned in \secref{sec:scientific_research}.
Therefore it was decided to go with a hybrid approach.