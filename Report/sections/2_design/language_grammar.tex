\subsection{Language Grammar} \label{langGram}

In this section we will describe how to formally specify a language using constructs such as Backus-Naur form (BNF), Extended Backus-Naur form(EBNF).
Following these two topics we will introduce the concept of derivation, and which type of derivation we have chosen for our language. 
Finally, we will give a justification for this choice as well as concrete examples of how all the previously mentioned elements are represented in our language.

\subsection{Backus-Naur form}

Backus-Naur form was a an evolution of a formal notation used to describe the language ALGOL 58, introduced by John Backus in 1959. The evolution happened as a result of modifications
made to this notation by Peter Naur in 1960, when he was describing ALGOL 60. Hence why the notation became known as Backus-Naur form (BNF).

As alluded to, BNF is used as a formal notation to describe the language, or more specifically its grammar. 


Short history of BNF
how its used
- Formal CFG..
examples


\subsection{Extended Backus-Naur form}


\subsection{Derivation}


\subsection{Justification}