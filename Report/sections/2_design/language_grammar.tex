\subsection{Language Grammar} \label{langGram}

In this section we will describe how to formally specify a language using constructs such as Backus-Naur form (BNF), Extended Backus-Naur form(EBNF).
Following these two topics we will introduce the concept of derivation, and which type of derivation we have chosen for our language.
Finally, we will give a justification for this choice as well as concrete examples of how all the previously mentioned elements are represented in our language.

\subsection{Backus-Naur form}

Backus-Naur form was a an evolution of a formal notation used to describe the language ALGOL 58, introduced by John Backus in 1959. The evolution happened as a result of modifications
made to this notation by Peter Naur in 1960, when he was describing ALGOL 60. Hence why the notation became known as Backus-Naur form (BNF)\cite{sebesta_concepts_2016}.

As alluded to, BNF is used as a formal notation to describe the language, or more specifically its grammar. 
The construction of this grammar is made up of a collection of production rules that are formed as shown in \figref{BNF_Example}.

\fig{0.4}{BNF_Example}{An example of a simple grammar written in BNF\cite{sebesta_concepts_2016}.}

In \figref{BNF_Example} there is a left hand side (LHS) and a right hand side (RHS), separated by an arrow. The LHS is a variable, also called the abstraction, that is being defined and the RHS is the definition.
This definition on the RHS consists of non-terminals and terminals. A non-terminal is a variable (sometimes called a symbol) that is an abstraction that have references to other abstractions. 
These non-terminal variables can have multiple distinct definitions each being a possible form of the syntax. 
If a non-terminal has multiple definitions these can either be represented as multiple rules as seen in \figref{BNF_Example} or as a single rule where each definition is seperated by an OR operator: '|'. 
Terminals correlate to the outputs of the production rules of the grammar, as shown in figure \figref{BNF_Example_Terminals} where the terminals of the language are highlighted in red. 
As the name implies, terminals cannot be derrived further. Terminals are also referred to as lexemes and tokens\cite{sebesta_concepts_2016}. 

\fig{0.4}{BNF_Example_Terminals}{The terminal symbols in the grammar from \figref{BNF_Example} highlighted in red\cite{sebesta_concepts_2016}.}

\subsection{Extended Backus-Naur form}

BNF has seen several different extensions.
These extended versions are commonly referred to as Extended BNF or just EBNF.
EBNF is not any more powerful than BNF, rather it simply adjusts and adds to the syntax in an attempt to make it more readable.
Whether EBNF succeeds in that regard is completely subjective.

Most versions of ENBF usually include three extensions:
- The use of square brackets to denote and optional part of the right-hand side of a production
- The use of curly brackets to indicate that the enclosed part may be iterated any number of times
- The use of the OR operator, |, to denote multiple-choice options for a production

Despite there being a standardized version of EBNF, it is rarely ever used.

Whichever version a language designer decides to use is up to preference.

\subsection{Derivation}


\subsection{Justification}
