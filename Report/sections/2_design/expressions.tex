\subsection{Expressions}
A major part of any programming language is expressions of which there are several types.
In the following, we will introduce these and discuss their relevance to \dazel{}.

\subsubsection*{Arithmetic expressions}
Arithmetic expressions are basic mathematical expressions. 
They consist of operators, operands, and function calls. 
Operators can either be unary, binary or ternary meaning they have either one, two or three operands.
Finally, binary operators are usually infix, meaning they appear between operands.

In \dazel{}, tiles on screens are placed on a grid with the bottom left corner being located at (0, 0) and only growing positively.
Because of this, the language is only concerned with positive numbers, and consequently it exclusively features a variety of binary operators. 

- arithmetic Expressions
- operator evaluation order
	- precedence
	- associativity
	- parantheses
	- side effects
- relational expressions
- boolean expressions
