\subsection{Expressions}
A major part of any programming language is expressions of which there are several types.
In the following, we will introduce these and discuss their relevance to \dazel{}.

\subsubsection*{Arithmetic expressions}
Arithmetic expressions are basic mathematical expressions.
They consist of operators, operands, and function calls. 
Operators can either be \textit{unary}, \textit{binary} or \textit{ternary} meaning they have either one, two or three operands.
Finally, binary operators are usually \textit{infix}, meaning they appear between operands.

\subsubsection*{Operator evaluation order}
The \textit{precedence} of an operator along with the \textit{associativity} rules in a language dictate the order in which operators are evaluated.
Different languages implement this differently. Some languages simply evaluate operators in the order that they appear. Usually, however, languages follow the order of operations defined in mathematics such as multiplication being of higher priority than addition.
Parentheses are also often part of expressions in programming languages and typically impacts the evaluation order of operators such that the contents of the parentheses are evaluated first.
Associativity in languages is most commonly implemented as left to right, however some operators like the exponentation operator sometimes associates right to left. 

\subsubsection*{Relational expressions}


\subsubsection*{Boolean expressions}

\subsubsection*{Expressions in \dazel{}}
\dazel{} features a variety of binary operators to support arithmetic as well as unary operations for boolean expressions.

- arithmetic Expressions
- operator evaluation order
	- precedence
	- associativity
	- parantheses
	- side effects
- relational expressions
- boolean expressions
