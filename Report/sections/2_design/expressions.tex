\subsection{Expressions}
A major part of any programming language is expressions of which there are several types\cite{sebesta_concepts_2016}.
In the following, we will introduce these and discuss their relevance to \dazel{}.

\subsubsection*{Arithmetic expressions}
Arithmetic expressions are basic mathematical expressions.
They consist of operators, operands, and function calls. 
Operators can either be \textit{unary}, \textit{binary} or \textit{ternary} meaning they have either one, two or three operands.
Finally, binary operators are usually \textit{infix}, meaning they appear between operands\cite{sebesta_concepts_2016}.

\subsubsection*{Operator evaluation order}
The \textit{precedence} of an operator along with the \textit{associativity} rules in a language dictate the order in which operators are evaluated.
Different languages implement this differently. Some languages simply evaluate operators in the order that they appear. Usually, however, languages follow the order of operations defined in mathematics such as multiplication being of higher priority than addition\cite{sebesta_concepts_2016}.


Parentheses are also often part of expressions in programming languages and typically impacts the evaluation order of operators such that the contents of the parentheses are evaluated first\cite{sebesta_concepts_2016}.


Associativity in languages is most commonly implemented as left to right, however some operators like the exponentation operator sometimes associates right to left\cite{sebesta_concepts_2016}. 

\subsubsection*{Relational expressions}
A relational expression is an expression that has two operands and one relational operator. The relational operator compares the values of the two operands and returns a
boolean, if booleans are included in the language. Relational operators include symbols such as \texttt{LESS THAN} and \texttt{LARGER THAN}, as well as \texttt{EQUAL TO} and
\texttt{NOT EQUAL TO}. 
Relational operators will typically have lower precedence than arithmetic operators to ensure proper evaluation\cite{sebesta_concepts_2016}.


\subsubsection*{Boolean expressions}
Boolean expressions are constructed using different mathematical logic operators such as the logical \texttt{AND}, \texttt{OR} and \texttt{EQUALS}. 
When evaluated, a whole expression is either true or false. 
This is then often combined with statements such as \texttt{if-then-else} statements to achieve a conditional program flow - if the expression evaluates to true, the code in the \texttt{then} block is executed, and if it evaluates to false, code in the \texttt{else} block is executed instead\cite{sebesta_concepts_2016}.

\subsubsection*{Expressions in \dazel{}}
\dazel{} features a variety of binary operators to support arithmetic as well as unary operations for boolean expressions.
It also has an operator evalation order identical to mathematics, as well as a left to right associativity.  
We thought that this was the most intuitive approach to prevent confusion for beginner programmers, and also maintains the standard that most experienced programmers are accustomed to.
To accomodate evalation of expressions \dazel{} also has the most common relational and logic operators as can be seen in \tabref{table:operatorsymbols}. 
