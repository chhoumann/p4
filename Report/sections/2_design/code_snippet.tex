\section{Code examples}

\begin{lstlisting}[caption={Example screen.}, label={lst:SampleScreen1},escapechar=|]
// ./Screens/SampleScreen1.txt
Screen SampleScreen1 
{
    Map 
    {
        Size(30, 24);

        Walls("Stone.png"); 
        Floor("Grass.jpg");

        Line(TopLeft, TopRight, "Cliff.png");
        Line([2, 2], [2, 4], "Cliff.png");

        Square(position, size, "Cliff.png");
        Square([8, 8], 4, "Cliff.png");
    }

    OnScreenEntered
    {
        if Player.KilledEntity(Entity.Skeleton1) 
        {
            // Player has killed a specific entity so a new exit appears
            SetTile([4, 0], "Stair.png");
            Exit([4, 0], SampleScreen3.Exit1);
        }
    }

    OnUpdate
    {
        if Player.Position == [4, 4] 
        {
            // Player is standing on a specific tile so a new exit appears
            SetTile([4, 5], "Stair.png");
            Exit([4, 4], SampleScreen4.Exit1);
        }
    }

    Entities
    {
        SpawnEntity(Skeleton1, [4, 4])
    }
    
    Exits 
    {
        // Two different ways to create an exit
        // Specific tile(s) or whole screen side
        Exit1 = Exit([4, 0], SampleScreen2.Exit1);
        ScreenExit(Bottom, SampleScreen2);
    }
}
\end{lstlisting}

Code \snipref{lst:SampleScreen1} shows an example of a screen. A screen represents a single area within the game, and each of the functions used within the \texttt{Screen} block define the layout, what happens when the screen is loaded, the entities that should be present as well as the exits of the screen. 

When a ScreenExit is created, one must first define the direction of the exit, for example the bottom of the screen as shown on line 36.
In the second actual parameter, one must then define the screen which the exit should connect to.
Then, once the player leaves the screen in the defined direction, the connected screen will load. An example of multiple screens that are interconnected is illustrated in \figref{Screen_Example_Layout}.

\fig{0.7}{Screen_Example_Layout}{Different screens can be interconnected via the \texttt{Exits} block shown in \snipref{lst:SampleScreen1}. Note that screens can have different sizes defined by the \texttt{Size} function shown on line 6 in the code snippet.}

The definition of each built-in function is shown in \tabref{table:explanations_of_functions}.