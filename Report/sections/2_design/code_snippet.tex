\section{Code examples}

\begin{lstlisting}[caption={Example screen}, label={lst:SampleScreen1},escapechar=|]
// ./Screens/SampleScreen1.txt
Screen SampleScreen1 
{
    Map 
    {
        Size(30, 24);

        Walls(Stone); 
        Floor(Grass);

        Line(TopLeft, TopRight, Cliff);
        Line([2, 2], [2, 4], Cliff);

        Square(position, size, Cliff);
        Square([8, 8], 4, Cliff);
    }

    OnScreenEntered
    {
        if Player.CompletedQuest(...) 
        {
            SetTile([4, 0], Stair);
        }
    }

    Entities
    {
        SpawnEntity(Skeleton1, [4, 4]).SetMovePattern(Square1);
    }
    
    Exits 
    {
        // Two different ways to create an exit
        // Specific tile(s) or whole screen side
        Exit1 = Exit([4, 0], SampleScreen2.Exit1);
        ScreenExit(Bottom, SampleScreen2);
    }
}
\end{lstlisting}

Code snippet~\ref{lst:SampleScreen1} shows an example of a screen. A screen represents a single area within the game, and each of the functions used within the Screen
define the layout, what happens as the screen is loaded, the entities that should be present as well as the exits of the level.
Objects such as Skeleton1 and the movement pattern Square1 on line 28 will be defined in a separate file.
The purpose of defining these separetely is to allow ease of reuseability and have a clear separation of concerns.