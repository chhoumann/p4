\chapter{Future Works}

In the following section, we will be discussing the aspects of the compiler and the interpreter that we would have liked to implement, had time not been the limiting factor. In order to discuss each aspect in a meaningful context the section will be subdivided into two - a section related to the language and the compiler and another related to the interpreter.

\subsection*{Completeness of the language}
As part of the ambition of making \dazel{} a good introduction to programming, the future features of the language should include the following: 

\begin{itemize}
    \item
      \textbf{Control flow} features such as loops and conditional statements with boolean expressions. The benefit of having control flow features such as these is that it would allow a user of \dazel{} to place objects on a map in a more elegant fashion without repeating code. An example of such a situation would be if one wishes to place walls along the edges of the screen or a situation where one wishes to have an entity appear on the screen based on some condition. These features would both improve writeability and readability.
    \item
      \textbf{Event callbacks} that execute a piece of code based when an event happens in the game. This ties into the aforementioned control flow features as this would allow you to execute conditional code based on specific circumstances such as when the player enters a screen. 
    \item
      \textbf{User defined functions} would allow reusability of code. 
    \item
      \textbf{Member access features including:}
      \begin{itemize}
        \item
            The ability to reference different type of variables such as integers and floats.
        \item
            The ability to use member access in expressions.    
        \item
            Member access properly searching for variables in symbol tables with the provided path through scopes instead of recursively looking for the first occurence of a variable with the given identifier. This would result in the expected behavior of the member access feature and therefore result in better writeability.
      \end{itemize}
    \item
      \textbf{Extend logging functionality} such that the \texttt{Print} function is able to be invoked with more than values and variables. For example, attempting to invoke \texttt{Print} with an expression results in an error rather than evaluating the expression and logging the result.
\end{itemize}

While the current speed of the compiler is satisfactory, there are still different aspects of its implementation we feel could be optimized and improved upon. 
For example, instead of rebuilding the entire AST when a change is made to a single \textbf{GameObject}, it would improve the compilation speed to only update that specific \textbf{GameObject} in the AST.

\subsection*{In-game features}
For the game itself, we came up with many ideas during the language design phase that we have not had the time to implement. 
\begin{itemize}
  \item
    \textbf{Movement and animations for entities} was among the most most important of these. 
    Currently, entity behavior defined inside entity \textbf{GameObjects} is simply parsed and accepted by the compiler, though it remains unused by the interpreter. 
  \item
      \textbf{More customizability} such as allowing the user to change the player graphics as mentioned in \secref{sec:Interpreter} was also discussed.
  \item 
    \textbf{The generation of tile exits and dungeons} is something we have wanted to include since the beginning. Tile exit models are generated by the compiler, however they remain unused by the interpreter. In addition, useful dungeon creation functions such as the \texttt{Walls} functions was not implemented in the game.
  \item
    \textbf{Dialogue with NPCs and quests} is another common aspect of adventure games, both of which we wanted to include to give meaningful progression through the game rather than only allowing exploration of pre-defined screens. 
  \item 
    \textbf{Player inventory with weapons and items} was another feature we felt would contribute greatly to giving the player a reason to explore the world to find new loot.
\end{itemize}

It is worth noting that the game features were de-emphasized in favor of the language features which is why there are more features missing from the in-game side of the project. There are likely more features that one could consider implementing, however the aforementioned points encompass what we felt would be the most important features to implement in the future for \dazel{}.