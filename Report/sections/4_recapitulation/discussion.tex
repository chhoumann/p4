\chapter{Discussion}
Our goal with this project was to design and implement a compiler for a programming language for beginners in the context of game development. 
This sections seeks to communicate the circumstances under which the project was developed as well as give insight into our thoughts about the end result.


\section{Ambitiousness}
A major challenge in the first half of the project that ended up impacting the project as a whole was that our original group had a difference in ambition. 
One half of the group had lower expectations which caused conflicts and limited the group work.
Much time was spent attemping to solve these conflicts and getting everyone to agree on how much effort and time to invest in the project.
This combined with the overall level of ambition meant that we fell behind as not much work was being done on the project for the first two months. 
Eventually, this meant that we decided to split our group in two halves as we were unable to find an agreeable compromise.

Because we stuck to the original vision for the project despite having less time and a smaller group, we unfortunately ended up investing too much time into features that we realistically did not have time to implement such as member access.
Features like this and others were not part of the semester curriculum which therefore meant that we had no frame of reference to work from.

Furthermore, the consequence of the group split meant that we were unable to spend as much time as we would have liked on the game aspect of \dazel{}. 
Thus we ended with a relatively simple prototype, however we feel that, under the circumstances, this is a satisfying proof of concept.

\section{Learning Goals}
- Frontend tools and learning
	- "udenfor pensum" i.e. member access, link checking and semantics for OOP principles
- Backend
	- We do not compile to bytecode, assembly, etc. - more to learn
	
\section{Criteria evalutation}
The new

\section{Testing}