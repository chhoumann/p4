\chapter{Importance of Programming} \label{chap:importance_of_programming}

As the humans have entered a digital age, more and more stuff gets 
digitalized every year. This transition has been ongoing for a few 
generations, and the need for software cannot be understated. The 
world runs on software, your coffee machine, television, car, mobile
phone, and the list goes on, not to mention that every single business 
and organization in the world needs an app, website, etc. Most companies
need custom made applications or websites, and therefore a lot of coders
is needed.

This digital transition has made humans more dependant on technology than ever - 
and for a good reason. The year 2020 when the pandemic hit the world is a good example of that.
The pandemic changed the every day lives of a lot of people. Due to restrictions and curfew,
people from all over the world had to stay home. This also applied to people 
who were studying or working as well. Business still had work that needed to be done,
students still had education they needed to finish. However, the world could not simply 
standby and wait for the pandemic to go over. 

In order to solve this problem,
various digital solutions have been used to prevent the world economy from receding.
Technology in form of video communication platforms such as Zoom and Microsoft Teams,
have allowed students to be able to stll receive teaching/lectures and study material online. 
Workers, who's work often consists of meetings and communication with other people,
have also been able to continue doing there work from home through video communication platforms. 
Restaurants and other businesses in the food industry have been able keep selling food, 
even with restrictions prohibiting customers to enjoy their food at the restaurant, 
they have shifted their sales to take away, where customers for example can order
food from their website or through an app such as Wolt. \newline


What do all these digital solutions have in common? They have all been created by 
some people programming them. Technology can help humans ease or solve these problems, 
but only if the machine or device have been programmed how to solve these problems. 
Zoom for example, would not have been able to allow people to have virtual meetings, 
if it had not been programmed to record each users webcam and microphone, and then output
it to each others devices. Wolt/restaurants could not have been able to manage
their orders online as effectively, if it was not for every step in the order process
being programmed - from displaying the restaurant's menu cart in the app, to make customers
able to order items from the menu and then send this information to Wolt workers that then 
deliver the orders. \newline

The takeaway from this is not that the the relevance of technology and programming 
only is due to the unexpected pandemic that occured,
but rather that the world and humans have - and will continue to face many problems post pandemic as well. Whether it is about gathering and analyzing data,
creating and improving automation such as in the production industry or even autonomous vechicles,
humans dependence and request for technology tbat ease their lives, will only continue
to grow and thus the importance of programming can be said to more important than ever.







