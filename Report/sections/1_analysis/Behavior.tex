\chapter{Behavior}\label{chap:behavior}

As concluded in the last chapter, learning how to program for the first time is hard.
Therefore, a natural next question would be to ask how one could make it easier.
That is the topic this chapter will examine. \\

Earlier, in \chapref{chap:analysis}, we presented \textit{deliberate practice} as an optimal way to learn.
However, merely knowing how to learn is not the same as learning. \\

For any behavior to occur, three elements must be present. Namely, motivation, ability to perform a given behavior, and a prompt \cite{bjfoggBehaviorModelPersuasive2009}. If any of those are not present, the behavior will not occur. \\

Often, increasing a user's ability is the best choice to optimizing behavior performance \cite{bjfoggBehaviorModelPersuasive2009}. Therefore, we should aim to make it simpler for users to program games.

Specifically, we will focus on the brain cycle aspect of learning how to program. If the user has to think hard to perform the behavior, it is not simple \cite{bjfoggBehaviorModelPersuasive2009}.

So the question becomes: how do we make it easier for users to program games?

% I assume our argument for choosing games is already in order.
% If not, it can be presented in this chapter without loss of comprehension.