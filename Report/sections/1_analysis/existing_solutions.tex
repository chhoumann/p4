\section{Existing solutions}\label{sec:existing_solutions}
There are many services and solutions that, in their own way, attempt to teach new programmers how to program.
We will attempt to get an overview of some of the more popular ones.

\subsection{Scratch}\label{ScratchSection}
\begin{quote}
    ``\emph{A key goal of Scratch is to introduce programming to those with no previous programming experience.}'' - Maloney et al, 2010\cite{maloneyScratchProgrammingLanguage2010}
\end{quote}

Scratch\cite{ScratchImagineProgram} is the first programming language that comes to mind - and for good reason. At the time of writing, Scratch has 68 million registered users\cite{ScratchImagineProgramStatistics}. This could indicate that there is a desire to learn how to program through a largely visual and interactive platform.

There are a few features which makes Scratch stand out as a platform to learn programming. Some of these are highlighted below.
\begin{itemize}
    \item Scratch is always live. A user is not restricted from editing while their script is running. This helps users stay engaged in ``testing, debugging, and improving their projects''
    \item Scratch provides visual feedback to their users
    \item Scratch does not show error messages. Instead, it attempts error-recovery
    \item Scratch's syntax consists of four types of blocks; command blocks, function blocks, trigger blocks, and control structure blocks
    \item There are only three data types; boolean, number, and string
    \item Concurrency, or multi-threading, is supported. A sprite can do several things at once
\end{itemize}\cite{maloneyScratchProgrammingLanguage2010}

\subsection{CodeCombat}\label{CodeCombatSection}
CodeCombat\cite{CodeCombatCodingGames} is similar to Scratch in that it aims to teach users how to program.
Their approaches are very different, though. CodeCombat is a programming game while Scratch is a programming language.

In CodeCombat, users are able to write code in JavaScript or Python\cite{CodeCombatCodingGames}. The user plays through a code-editor, writing lines of code which moves their character. It is similar to accessing their character through an API.

In CodeCombat, you learn about various programming concepts such as syntax, methods, parameters, loops, and variables through the levels in the game\cite{CodeCombatCodingGames}.

\subsection{Conclusion}
Both \secref{CodeCombatSection} and \secref{ScratchSection} present existing solutions that are designed to introduce users to programming.

CodeCombat provides a representation of programming that is closer to writing code than Scratch, whereas Scratch provides blocks as an abstraction over textual code.
Another difference is how users are taught programming.
Using Scratch, learning happens implicitly.
You are taught to be creative through code. There are tutorials on how to do specific things, but no specific problems to solve.
CodeCombat, on the other hand, is more explicit in teaching the user programming concepts.
Learning is necessary in order to progress through the levels. You are taught to solve problems through code.

While the services have different approaches, they have a few important similarities.
For one, users are urged to learn through practice.
In Scratch, you get to create whatever your imagination allows. You are taught programming through creation and exploration.
CodeCombat teaches you programming through games; specific problems and obstacles that can be solved by putting what you have learned to practice.

Both services are also primarily visual.
There is a direct connection between the written code and what the characters do.