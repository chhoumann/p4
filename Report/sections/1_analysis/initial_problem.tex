\section{Initial problem}\label{sec:initial_problem}
In the introduction to \chapref{chap:analysis} we asked whether it would be possible to design a simple language which would act as a bridge to programming for beginners.
Having examined some existing solutions to how beginners can be taught programming, a more specific problem can now be formulated.

In the simple programming language, inspiration should be drawn from what works in Scratch and CodeCombat. Furthermore, the language should serve as an attempt to both breathe fresh air onto the learning experience for beginners in addition to solving some of the problems that new programmers face.

Below is the proposed intial problem statement.

\begin{figure}[h]
    \vspace{0.5cm}
    \centering
    \begin{framed}
        A programming language with a simple syntax which resembles that of modern programming languages while promoting practical learning and providing visual feedback to the written code.
    \end{framed}
    \vspace{-0.5cm}
    \label{fig:dazel_specification}
    \vspace{0.5cm}
\end{figure}

Henceforth, this language will be refered to as the \dazel{} programming language.
To create a language like \dazel{}, the language design must attempt to accommodate the specification as much as possible. \Secref{sec:scientific_research} presents research which will guide the design of \dazel{}.