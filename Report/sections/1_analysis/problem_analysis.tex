\chapter{Problem Analysis} \label{chap:analysis}
Learning how to code is hard.
It is no wonder that the most viewed video on YouTube for the search term '\textit{learn programming}' has been viewed 23 million times\cite{LearnProgrammingYouTube}.
The video concerns itself with \textit{Python}, a language that is regarded as an easy to learn for beginners\cite{PythonBeginners}.

A common error, and source of frustration, for new programmers is syntax errors.
Other issues include types and confusing error outputs \cite{bosseWhyProgrammingDifficult2017}.
New programmers also have difficulties with the application of programming concepts.
This could be because learning isn't practical enough\cite{lahtinenStudyDifficultiesNovice2005}.
You have to program to learn how to program.

Is a simple language, like Python\cite{WhatPythonExecutive}, the solution for new programmers?
Research indicates that it might be. A simple language gives rise to fewer syntax errors as well as fewer logic errors.
Furthermore, learning a simpler language does not seem to hinder the learner when moving to a more complex language\cite{mannilaWhatSimpleLanguage2006}.


Knowing this, can a simple language be created which acts as a bridge to programming for beginners?
Before a more specific problem can be identified, we will get an overview of some existing solutions to this problem.

\section{Similar Solutions}
% ! See Notion -> Outline. We should have defined an initial problem which we can refer to. Just like I do below.
This section aims to give an overview of how other technologies attempt to solve a problem similar to ours. We decided to research these similar solutions such that we can draw inspiration from them.

\subsection{Scratch}\label{ScratchSection}
\begin{quote}
    ``\emph{A key goal of Scratch is to introduce programming to those with no previous programming experience.}'' - Maloney et al, 2010\cite{maloneyScratchProgrammingLanguage2010}
\end{quote}

Scratch\cite{ScratchImagineProgram} is the first programming language that comes to mind - and for good reason. At the time of writing, Scratch has 68 million registered users\cite{ScratchImagineProgramStatistics}. This makes it the most popular solution for a closely adjacent problem to ours. It also indicates that there is a desire to learn how to program through a largely visual and interactive platform.


% !! I have chosen to expand exhaustively on what makes Scratch stand out. It is, by far, the most popular and well-developed platform. They also provided a paper, which made it much more scientifically valid for us to use. There is much to learn from it. If we do not draw inspiration from any of the below features, we could (and perhaps, should) make this a bit shorter.
There are a few features which makes Scratch stand out as a platform to learn programming.
\begin{itemize}
    \item Scratch is always live. A user is not restricted from editing while their script is running. This helps users stay engaged in ``testing, debugging, and improving their projects.''\cite{maloneyScratchProgrammingLanguage2010}
    \item Scratch provides visual feedback to their users.\cite{maloneyScratchProgrammingLanguage2010}
    \item Scratch does not show error messages. Instead, it attempts error-recovery.\cite{maloneyScratchProgrammingLanguage2010}
    \item Scratchs' syntax consists of four types of blocks; command blocks, function blocks, trigger blocks, and control structure blocks.\cite{maloneyScratchProgrammingLanguage2010}
    \item There are only three data types; boolean, number, and string.\cite{maloneyScratchProgrammingLanguage2010}
    \item Concurrency, or multi-threading, is supported. A sprite can several things at once.\cite{maloneyScratchProgrammingLanguage2010}
\end{itemize}

\subsection{CodeCombat}\label{CodeCombatSection}
CodeCombat is similar to Scratch in that it aims to teach users how to program.
Their approaches are very different, though. CodeCombat is a programming game while Scratch is a programming language.

In CodeCombat, users are able to write code in JavaScript or Python\cite{CodeCombatCodingGames}. The user plays through a code-editor, writing lines of code which moves their character. It is similar to accessing their character through an API.

In CodeCombat, you learn about various programming concepts such as syntax, methods, parameters, loops, and variables through the levels in the game\cite{CodeCombatCodingGames}.

\subsection{Conclusion}
Both \secref{CodeCombatSection} and \secref{ScratchSection} present existing solutions that are designed to introduce users to programming.

They provide different approaches in order to achieve this. CodeCombat provides a representation of programming that is closer to writing code than Scratch, as Scratch provides blocks as an abstraction over textual code.
Another difference is how users are taught programming. Using Scratch, learning happens implicitly. There are no levels designed to teach you programming concepts - but there are tutorials. The primary focus is on building. CodeCombat, on the other hand, is more explicit in teaching the user programming concepts. Learning is necessary in order to progress through the levels.

% ! Should this section be rounded off more? I have given a generalized conclusion as this section should come before the problem definition. We can draw upon inspiration from these existing solutions there.
% ! I guess you could say that we draw inspiration from both. We, like Scratch, focus on world- and story-creation, which gives more freedom than the linear story of CodeCombat. We draw from CodeCombat in that we provide a textual way of creating.
% !! On the background of the existing solutions, we can draw some conclusions and narrow the problem down...
\section{Initial problem}\label{sec:initial_problem}
In the introduction to \chapref{chap:analysis} we asked whether it would be possible to design a simple language which would act as a bridge to programming for beginners.
Having examined some existing solutions to how beginners can be taught programming, a more specific problem can now be formulated.

In the simple programming language, inspiration should be drawn from what works in Scratch and CodeCombat. Furthermore, the language should serve as an attempt to both breathe fresh air onto the learning experience for beginners in addition to solving some of the problems that new programmers face.

Below is the proposed specification of such a language.

\begin{figure}[h]
    \vspace{0.5cm}
    \centering
    \begin{framed}
        A programming language with a simple syntax which resembles that of modern programming languages while promoting practical learning and providing visual feedback to the written code.
    \end{framed}
    \vspace{-0.5cm}
    \caption{Specification for the \dazel{} programming language}
    \label{fig:dazel_specification}
    \vspace{0.5cm}
\end{figure}

To create a language like \dazel{}, the language design must attempt to accommodate the specification as much as possible. \Secref{sec:scientific_research} presents research which will guide the design of \dazel{}.
% !! Now that the initial problem has been specified, we can bridge towards design...
\section{Studies on intuitive language design} \label{sec:scientific_research}
In this section, we discuss relevant scientific research on the difficulties of programming for beginners.
Later, we shall use this as a basis for various design decisions in \dazel{} such that we can make more
informed decisions about the language syntax and semantics.

In an investigation by Stefik et al.\cite{stefik_empirical_2011}, participants - all of whom were
programming novices - would use different programming languages to solve the same tasks so as to compare intuitiveness.
It was concluded that keywords having names that directly alluded to their purpose was important to the novices.
For example, the keyword \emph{repeat} was rated to be nearly seven times more intuitive to use than the keyword \emph{for}.
As language designers, such discoveries are important to take note of, as it emphasizes the importance of using intuitive words for language keywords.

A different investigation from 2019 about children and programming by Papavlasopoulou et al.\cite{papavlasopoulou_exploring_2019} discovered that
programming stereotypes exist, even among children.
In the study, it was found that many of the participating girls had no experience with programming, and that they believed it
to be an activity exclusively performed by "geeks".
To remedy this prejudice, the authors had the children develop storyboards in groups from which they would develop games using the
Scratch programming language.
This made the girls realize the potential of storytelling by programming video games, and it helped motivate them to learn programming.

This study highlights the importance of presenting not just the mathematical aspect of programming to novices, but also
showing them other uses, such as being able to develop entertainment.
It also shows that being able to tell a story makes learning programming more engaging for beginners.


A final investigation we would like to discuss is by Bosse et al.\cite{bosseWhyProgrammingDifficult2017}.
It followed different progamming courses and gathered data from the participants' diaries as well as through interviews with the teachers.
This investigation found that variable typing was quite troublesome for the participants, as they had trouble understanding when to use
types such as \texttt{double}, \texttt{float} or \texttt{int} to represent numbers. This also led to issues when using division, as participants
did not understand why the expression 1/2 yielded the integer 0 when they expected 0.5.
It was also found that functions and return types were difficult to grasp for the course participants, implying that basic mathematical
knowledge is an important prerequisite for learning programming, as both concepts are a major part of both mathematics and science in general.
Finally, the participants also had trouble understanding scope rules, which may suggest that language designers should consider how to be more explicit about scope rules in their language.


When we design our language in \chapref{chap:language_design}, we will use the conclusions from these studies to help us design a language that will suit our target demographic
as best as possible.
